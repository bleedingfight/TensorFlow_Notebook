\documentclass{article}
\usepackage[space]{ctex}
\usepackage{graphicx}
\usepackage{minted}
\usepackage[margin=2.5cm]{geometry}
\usepackage{xcolor}
\usepackage{float}
\definecolor{bg}{rgb}{0.95,0.95,0.95}
\newmintinline{bash}{bgcolor=bg,linenos,breakanywhere,breaklines}
\newminted{bash}{bgcolor=bg,linenos,breakanywhere,breaklines}
\newmintinline{python}{bgcolor=bg,linenos,python3,breakanywhere,breaklines}
\newminted{python}{bgcolor=bg,linenos,python3,breakanywhere,breaklines}
\begin{document}
\section{The Python Fire Guide}

\subsection{Introduction}

Welcome to the Python Fire guide! Python Fire is a Python library that will turn any Python component into a command line interface with just a single call to Fire.

\subsection{Let's get started!}

\subsection{Installation}

To install Python Fire from pypi, run:

\pythoninline{pip install fire}

Alternatively, to install Python Fire from source, clone the source and run:

python setup.py install

Hello World

Version 1: fire.Fire()

The easiest way to use Fire is to take any Python program, and then simply call fire.Fire() at the end of the program. This will expose the full contents of the program to the command line.

import fire

def hello(name):
  return 'Hello {name}!'.format(name=name)

if __name__ == '__main__':
  fire.Fire()
Here's how we can run our program from the command line:

$ python example.py hello World
Hello World!
Version 2: fire.Fire(<fn>)

Let's modify our program slightly to only expose the hello function to the command line.

import fire

def hello(name):
  return 'Hello {name}!'.format(name=name)

if __name__ == '__main__':
  fire.Fire(hello)
Here's how we can run this from the command line:

$ python example.py World
Hello World!
Notice we no longer have to specify to run the hello function, because we called fire.Fire(hello).

Version 3: Using a main

We can alternatively write this program like this:

import fire

def hello(name):
  return 'Hello {name}!'.format(name=name)

def main():
  fire.Fire(hello)

if __name__ == '__main__':
  main()
Or if we're using entry points, then simply this:

import fire

def hello(name):
  return 'Hello {name}!'.format(name=name)

def main():
  fire.Fire(hello)
Exposing Multiple Commands

In the previous example, we exposed a single function to the command line. Now we'll look at ways of exposing multiple functions to the command line.

Version 1: fire.Fire()

The simplest way to expose multiple commands is to write multiple functions, and then call Fire.

import fire

def add(x, y):
  return x + y

def multiply(x, y):
  return x * y

if __name__ == '__main__':
  fire.Fire()
We can use this like so:

$ python example.py add 10 20
30
$ python example.py multiply 10 20
200
You'll notice that Fire correctly parsed 10 and 20 as numbers, rather than as strings. Read more about argument parsing here.

Version 2: fire.Fire(<dict>)

In version 1 we exposed all the program's functionality to the command line. By using a dict, we can selectively expose functions to the command line.

import fire

def add(x, y):
  return x + y

def multiply(x, y):
  return x * y

if __name__ == '__main__':
  fire.Fire({
      'add': add,
      'multiply': multiply,
  })
We can use this in the same way as before:

$ python example.py add 10 20
30
$ python example.py multiply 10 20
200
Version 3: fire.Fire(<object>)

Fire also works on objects, as in this variant. This is a good way to expose multiple commands.

import fire

class Calculator(object):

  def add(self, x, y):
    return x + y

  def multiply(self, x, y):
    return x * y

if __name__ == '__main__':
  calculator = Calculator()
  fire.Fire(calculator)
We can use this in the same way as before:

$ python example.py add 10 20
30
$ python example.py multiply 10 20
200
Version 4: fire.Fire(<class>)

Fire also works on classes. This is another good way to expose multiple commands.

import fire

class Calculator(object):

  def add(self, x, y):
    return x + y

  def multiply(self, x, y):
    return x * y

if __name__ == '__main__':
  fire.Fire(Calculator)
We can use this in the same way as before:

$ python example.py add 10 20
30
$ python example.py multiply 10 20
200
Why might you prefer a class over an object? One reason is that you can pass arguments for constructing the class too, as in this broken calculator example.

import fire

class BrokenCalculator(object):

  def __init__(self, offset=1):
      self._offset = offset

  def add(self, x, y):
    return x + y + self._offset

  def multiply(self, x, y):
    return x * y + self._offset

if __name__ == '__main__':
  fire.Fire(BrokenCalculator)
When you use a broken calculator, you get wrong answers:

$ python example.py add 10 20
31
$ python example.py multiply 10 20
201
But you can always fix it:

$ python example.py add 10 20 --offset=0
30
$ python example.py multiply 10 20 --offset=0
200
Unlike calling ordinary functions, which can be done both with positional arguments and named arguments (--flag syntax), arguments to __init__ functions must be passed with the --flag syntax. See the section on calling functions for more.

Grouping Commands

Here's an example of how you might make a command line interface with grouped commands.

class IngestionStage(object):

  def run(self):
    return 'Ingesting! Nom nom nom...'

class DigestionStage(object):

  def run(self, volume=1):
    return ' '.join(['Burp!'] * volume)

  def status(self):
    return 'Satiated.'

class Pipeline(object):

  def __init__(self):
    self.ingestion = IngestionStage()
    self.digestion = DigestionStage()

  def run(self):
    self.ingestion.run()
    self.digestion.run()

if __name__ == '__main__':
  fire.Fire(Pipeline)
Here's how this looks at the command line:

$ python example.py run
Ingesting! Nom nom nom...
Burp!
$ python example.py ingestion run
Ingesting! Nom nom nom...
$ python example.py digestion run
Burp!
$ python example.py digestion status
Satiated.
You can nest your commands in arbitrarily complex ways, if you're feeling grumpy or adventurous.

Accessing Properties

In the examples we've looked at so far, our invocations of python example.py have all run some function from the example program. In this example, we simply access a property.

from airports import airports

import fire

class Airport(object):

  def __init__(self, code):
    self.code = code
    self.name = dict(airports).get(self.code)
    self.city = self.name.split(',')[0] if self.name else None

if __name__ == '__main__':
  fire.Fire(Airport)
Now we can use this program to learn about airport codes!

$ python example.py --code=JFK code
JFK
$ python example.py --code=SJC name
San Jose-Sunnyvale-Santa Clara, CA - Norman Y. Mineta San Jose International (SJC)
$ python example.py --code=ALB city
Albany-Schenectady-Troy
By the way, you can find this airports module here.

Chaining Function Calls

When you run a Fire CLI, you can take all the same actions on the result of the call to Fire that you can take on the original object passed in.

For example, we can use our Airport CLI from the previous example like this:

$ python example.py --code=ALB city upper
ALBANY-SCHENECTADY-TROY
This works since upper is a method on all strings.

So, if you want to set up your functions to chain nicely, all you have to do is have a class whose methods return self. Here's an example.

import fire

class BinaryCanvas(object):
  """A canvas with which to make binary art, one bit at a time."""

  def __init__(self, size=10):
    self.pixels = [[0] * size for _ in range(size)]
    self._size = size
    self._row = 0  # The row of the cursor.
    self._col = 0  # The column of the cursor.

  def __str__(self):
    return '\n'.join(' '.join(str(pixel) for pixel in row) for row in self.pixels)

  def show(self):
    print(self)
    return self

  def move(self, row, col):
    self._row = row % self._size
    self._col = col % self._size
    return self

  def on(self):
    return self.set(1)

  def off(self):
    return self.set(0)

  def set(self, value):
    self.pixels[self._row][self._col] = value
    return self

if __name__ == '__main__':
  fire.Fire(BinaryCanvas)
Now we can draw stuff :).

$ python example.py move 3 3 on move 3 6 on move 6 3 on move 6 6 on move 7 4 on move 7 5 on __str__
0 0 0 0 0 0 0 0 0 0
0 0 0 0 0 0 0 0 0 0
0 0 0 0 0 0 0 0 0 0
0 0 0 1 0 0 1 0 0 0
0 0 0 0 0 0 0 0 0 0
0 0 0 0 0 0 0 0 0 0
0 0 0 1 0 0 1 0 0 0
0 0 0 0 1 1 0 0 0 0
0 0 0 0 0 0 0 0 0 0
0 0 0 0 0 0 0 0 0 0
It's supposed to be a smiley face.

Can we make an even simpler example than Hello World?

Yes, this program is even simpler than our original Hello World example.

import fire
english = 'Hello World'
spanish = 'Hola Mundo'
fire.Fire()
You can use it like this:

$ python example.py english
Hello World
$ python example.py spanish
Hola Mundo
Calling Functions

Arguments to a constructor are passed by name using flag syntax --name=value.

For example, consider this simple class:

import fire

class Building(object):

  def __init__(self, name, stories=1):
    self.name = name
    self.stories = 1

  def climb_stairs(self, stairs_per_story=10):
    for story in range(self.stories):
      for stair in range(1, stairs_per_story):
        yield stair
        yield 'Phew!'
    yield 'Done!'

if __name__ == '__main__':
  fire.Fire(Building)
We can instantiate it as follows: python example.py --name="Sherrerd Hall"

Arguments to other functions may be passed positionally or by name using flag syntax.

To instantiate a Building and then run the climb_stairs function, the following commands are all valid:

$ python example.py --name="Sherrerd Hall" --stories=3 climb_stairs 10
$ python example.py --name="Sherrerd Hall" climb_stairs --stairs_per_story=10
$ python example.py --name="Sherrerd Hall" climb_stairs --stairs-per-story 10
$ python example.py climb-stairs --stairs-per-story 10 --name="Sherrerd Hall"
You'll notice that hyphens and underscores (- and _) are interchangeable in member names and flag names.

You'll also notice that the constructor's arguments can come after the function's arguments or before the function.

You'll also notice that the equal sign between the flag name and its value is optional.

Functions with *varargs and **kwargs

Fire supports functions that take *varargs or **kwargs. Here's an example:

import fire

def order_by_length(*items):
  """Orders items by length, breaking ties alphabetically."""
  sorted_items = sorted(items, key=lambda item: (len(str(item)), str(item)))
  return ' '.join(sorted_items)

if __name__ == '__main__':
  fire.Fire(order_by_length)
To use it, we run:

$ python example.py dog cat elephant
cat dog elephant
You can use a separator to indicate that you're done providing arguments to a function. All arguments after the separator will be used to process the result of the function, rather than being passed to the function itself. The default separator is the hyphen -.

Here's an example where we use a separator.

$ python example.py dog cat elephant - upper
CAT DOG ELEPHANT
Without the separator, upper would have been treated as another argument.

$ python example.py dog cat elephant upper
cat dog upper elephant
You can change the separator with the --separator flag. Flags are always separated from your Fire command by an isolated --. Here's an example where we change the separator.

$ python example.py dog cat elephant X upper -- --separator=X
CAT DOG ELEPHANT
Separators can be useful when a function accepts *varargs, **kwargs, or default values that you don't want to specify. It is also important to remember to change the separator if you want to pass - as an argument.

Argument Parsing

The types of the arguments are determined by their values, rather than by the function signature where they're used. You can pass any Python literal from the command line: numbers, strings, tuples, lists, dictionaries, (sets are only supported in some versions of Python). You can also nest the collections arbitrarily as long as they only contain literals.

To demonstrate this, we'll make a small example program that tells us the type of any argument we give it:

import fire
fire.Fire(lambda obj: type(obj).__name__)
And we'll use it like so:

$ python example.py 10
int
$ python example.py 10.0
float
$ python example.py hello
str
$ python example.py '(1,2)'
tuple
$ python example.py [1,2]
list
$ python example.py True
bool
$ python example.py {name: David}
dict
You'll notice in that last example that bare-words are automatically replaced with strings.

Be careful with your quotes! If you want to pass the string "10", rather than the int 10, you'll need to either escape or quote your quotes. Otherwise Bash will eat your quotes and pass an unquoted 10 to your Python program, where Fire will interpret it as a number.

$ python example.py 10
int
$ python example.py "10"
int
$ python example.py '"10"'
str
$ python example.py "'10'"
str
$ python example.py \"10\"
str
Be careful with your quotes! Remember that Bash processes your arguments first, and then Fire parses the result of that. If you wanted to pass the dict {"name": "David Bieber"} to your program, you might try this:

$ python example.py '{"name": "David Bieber"}'  # Good! Do this.
dict
$ python example.py {"name":'"David Bieber"'}  # Okay.
dict
$ python example.py {"name":"David Bieber"}  # Wrong. This is parsed as a string.
str
$ python example.py {"name": "David Bieber"}  # Wrong. This isn't even treated as a single argument.
<error>
$ python example.py '{"name": "Justin Bieber"}'  # Wrong. This is not the Bieber you're looking for. (The syntax is fine though :))
dict
Boolean Arguments

The tokens True and False are parsed as boolean values.

You may also specify booleans via flag syntax --name and --noname, which set name to True and False respectively.

Continuing the previous example, we could run any of the following:

$ python example.py --obj=True
bool
$ python example.py --obj=False
bool
$ python example.py --obj
bool
$ python example.py --noobj
bool
Be careful with boolean flags! If a token other than another flag immediately follows a flag that's supposed to be a boolean, the flag will take on the value of the token rather than the boolean value. You can resolve this: by putting a separator after your last flag, by explicitly stating the value of the boolean flag (as in --obj=True), or by making sure there's another flag after any boolean flag argument.

Using Fire Flags

Fire CLIs all come with a number of flags. These flags should be separated from the Fire command by an isolated --. If there is at least one isolated -- argument, then arguments after the final isolated -- are treated as flags, whereas all arguments before the final isolated -- are considered part of the Fire command.

One useful flag is the --interactive flag. Use the --interactive flag on any CLI to enter a Python REPL with all the modules and variables used in the context where Fire was called already available to you for use. Other useful variables, such as the result of the Fire command will also be available. Use this feature like this: python example.py -- --interactive.

You can add the help flag to any command to see help and usage information. Fire incorporates your docstrings into the help and usage information that it generates. Fire will try to provide help even if you omit the isolated -- separating the flags from the Fire command, but may not always be able to, since help is a valid argument name. Use this feature like this: python example.py -- --help.

The complete set of flags available is shown below, in the reference section.

Reference

Setup	Command	Notes
install	pip install fire	
Creating a CLI

Creating a CLI	Command	Notes
import	import fire	
Call	fire.Fire()	Turns the current module into a Fire CLI.
Call	fire.Fire(component)	Turns component into a Fire CLI.
Flags

Using a CLI	Command	Notes
Help	command -- --help	Show help and usage information for the command.
REPL	command -- --interactive	Enter interactive mode.
Separator	command -- --separator=X	This sets the separator to X. The default separator is -.
Completion	command -- --completion	Generate a completion script for the CLI.
Trace	command -- --trace	Gets a Fire trace for the command.
Verbose	command -- --verbose	Include private members in the output.
Note that flags are separated from the Fire command by an isolated -- arg.
\end{document}
