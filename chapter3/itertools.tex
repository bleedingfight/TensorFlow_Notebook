\section{itertools}
Python的内建模块itertools提供了非常有用的用于操作迭代对象的函数。首先我们看itertools提供的无限迭代器。
\begin{python}
import itertools
natuals = itertools.count(1)
for i in natuals:
    print(i)
\end{python}
因为count()会创建一个无限的迭代其。所以上述代码会一直打印。

\subsection{cycle}
cycle()会把传入的一个序列无限的重复下去:
\begin{python}
import itertools
cs = itertools.cycle('ABC')
for i in cs:
    print(i)
\end{python}
无限重复打印'A','B','C'。

repeat()负责把一个元素无限重复下去,不过如果提供第二个参数就可以限定重复的次数:
\begin{python}
In [63]: ns = itertools.repeat('Ab',3)
In [64]: for i in ns:
    ...:     print(i)
        ...:     
	Ab
	Ab
	Ab
\end{python}
无限序列只有在for迭代时才会无限的迭代下去,如果只是创建一个迭代对象,他不会事先把无线个元素生成出来,是时尚也不可能在内存中创建无线多个元素/无线序列虽然可以无线迭代下去,但是通常我们会通过takewhile()等函数根据条件判断截取一个有限的序列:
\begin{python}
n [65]: natuals = itertools.count(2)
In [66]: ns = itertools.takewhile(lambda x:x<10,natuals)
In [68]: ns
Out[68]: <itertools.takewhile at 0x7f306c12fdc8>
In [69]: list(ns)
Out[69]: [2, 3, 4, 5, 6, 7, 8, 9]
\end{python}
\subsection{chain()}
chain()可以把一组迭代对象串联起来,形成一个更大的迭代器:
\begin{python}
In [71]: for c in itertools.chain('ABC','XYZ'):
    ...:     print(c)
	    ...:     
	    A
	    B
	    C
	    X
	    Y
	    Z

\end{pyhton}
\subsection{groupby()}
groupby()把迭代器中相邻的重复元素挑出来放在一起:
\begin{python}
[74]: for key,group in itertools.groupby('AAABBCCCAA'):
...:     print(key,list(group))
...:     
A ['A', 'A', 'A']
B ['B', 'B']
C ['C', 'C', 'C']
A ['A', 'A']
\end{python}
实际上挑选规则是通过函数完成的,只要作用于函数的两个元素返回的值相等,这两个元素就被认为是在一组,而函数返回值作为组的key。如果我们要忽略大小写分组,就可以让元素'A'和'a'都返回相同的key:
\begin{python}
In [76]: for key, group in itertools.groupby('AaaBBbcCAAa', lambda c: c.upper()):
            print(key, list(group))
	A ['A', 'a', 'a']
	B ['B', 'B', 'b']
	C ['c', 'C']
	A ['A', 'A', 'a']

\end{python}

