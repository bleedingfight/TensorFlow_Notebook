\subsection{batch normalization}
\begin{itemize}
	\item[\S] 数据x为Tensor。
\item mean:为x的均值,也是一个Tensor。
\item var:为x的方差,也为一个Tensor。
\item offset:一个偏移,也是一个Tensor。
\item scale:缩放倍数,也是一个Tensor。
\item variable\_epsilon,一个不为0的浮点数。
\item name:操作的名字,可选。
\end{itemize}
batch normalization计算方式是:
\begin{gather}
x = (x-\bar{x})/\sqrt{Var(x)+variable_{epsilon}}\\
x = x\times scale+offset\\
\end{gather}
\begin{gather}
\text{均值}:\bar{x} = \frac{1}{m}\Sigma_{i=1}^{m}x_i\\
\text{方差}:\sigma^2 = \frac{1}{m}\Sigma_{i=1}^m(x_i-\bar{x})
\end{gather}
\section{常见的激活函数}
\subsection{relu}
relu函数在自变量x小于0时值全为0,在x大于0时,值和自变量相等。
\begin{lstlisting}[language=Python]
import tensorflow as tf 
import matplotlib.pyplot as plt 
x = tf.linspace(-10.,10.,100)
y = tf.nn.relu(x)
with tf.Session() as sess:
	[x,y] = sess.run([x,y])
plt.plot(x,y,'r',6,6,'bo')
plt.title('relu')
ax = plt.gca()
ax.annotate("",
            xy=(6, 6), xycoords='data',
            xytext=(6, 4.5), textcoords='data',
            arrowprops=dict(arrowstyle="->",
                            connectionstyle="arc3"),
            )
ax.annotate("",xy=(6,6),xycoords='data',
            xytext=(10, 6), textcoords='data',
            arrowprops=dict(arrowstyle="->",
                            connectionstyle="arc3"),
	  	   
)
ax.grid(True)
plt.xlabel('x')
plt.ylabel('relu(x)')
plt.savefig('relu.png',dpi = 600)
\end{lstlisting}
\subsection{relu6}
relu6函数和relu不同之处在于在x大于等于6的部分值保持为6。
\begin{lstlisting}[language=Python]
import tensorflow as tf 
import matplotlib.pyplot as plt 
x = tf.linspace(-10.,10.,100)
y = tf.nn.relu6(x)
with tf.Session() as sess:
	[x,y] = sess.run([x,y])
plt.plot(x,y,'r',6,6,'bo')
plt.title('relu6')
ax = plt.gca()
ax.annotate("",
            xy=(6, 6), xycoords='data',
            xytext=(6, 4.5), textcoords='data',
            arrowprops=dict(arrowstyle="->",
                            connectionstyle="arc3"),
            )
ax.grid(True)
plt.xlabel('x')
plt.ylabel('relu6(x)')
plt.savefig('relu6.png',dpi = 600)
\end{lstlisting}
\begin{figure}[H]
\centering
\includegraphics[scale=0.8]{./pic/chapter1/relu.png}
\caption{relu}
\end{figure}
\begin{figure}[H]
\centering
\includegraphics[scale=0.8]{./pic/chapter1/relu6.png}
\caption{relu6}
\end{figure}

\subsection{sigmoid}
\begin{lstlisting}[language=Python]
import tensorflow as tf 
import matplotlib.pyplot as plt 
import matplotlib.patches as mpatches
x = tf.linspace(-10.,10.,100)
y1 = tf.nn.sigmoid(x)
y2 = tf.nn.tanh(x)
red_patch = mpatches.Patch(color = 'red',label = 'sigmoid')
blue_patch = mpatches.Patch(color = 'blue',label = 'tanh')
with tf.Session() as sess:
	[x,y1,y2] = sess.run([x,y1,y2])
plt.plot(x,y1,'r',x,y2,'b')
ax = plt.gca()
ax.annotate(r"$tanh(x) = \frac{1-^{-2x}}{1+e^{-x}}$",
	   xy=(0,0),xycoords="data",
	   xytext=(1,0),textcoords="data",
	   arrowprops=dict(arrowstyle="->",
	   connectionstyle="arc3"),
)
ax.annotate(r"$sigmoid(x) = \frac{1}{1+e^{-x}}$",
	   xy=(0,0.5),xycoords="data",
	   xytext=(1,0.5),textcoords="data",
	   arrowprops=dict(arrowstyle="->",
	   connectionstyle="arc3"),
)
plt.xlabel('x')
plt.grid(True)
plt.legend(handles = [red\_patch,blue\_patch])
plt.savefig('activate.png',dpi=600)
\end{lstlisting}
\begin{figure}[H]
\centering
\includegraphics[scale=0.8]{./pic/chapter1/activate_fun.png}
\caption{activate\_fun}
\end{figure}
\subsection{relu和softplus}
\begin{lstlisting}[language=Python]
import tensorflow as tf 
import matplotlib.pyplot as plt 
import matplotlib.patches as mpatches
x = tf.linspace(-10.,10.,100)
y2 = tf.nn.softplus(x)
y3 = tf.nn.relu(x)
blue_patch = mpatches.Patch(color = 'blue',label = 'softplus')
yellow_patch = mpatches.Patch(color = 'yellow',label = 'relu')
with tf.Session() as sess:
	[x,y2,y3] = sess.run([x,y2,y3])
plt.plot(x,y2,'b',x,y3,'y')
ax = plt.gca()
plt.xlabel('x')
ax.annotate(r"$softplus(x)=log(1+e^x)$",
	   xy=(0,0),xycoords="data",
	   xytext=(1,0),textcoords="data",
	   arrowprops=dict(arrowstyle="->",
	   connectionstyle="arc3"),
)
ax.annotate(r"$relu(x)=max(x,0)$",
	   xy=(0,0.5),xycoords="data",
	   xytext=(1,0.5),textcoords="data",
	   arrowprops=dict(arrowstyle="->",
	   connectionstyle="arc3"),
)

plt.grid(True)
plt.legend(handles = [blue_patch,yellow_patch])
plt.savefig('relu_softplus.png',dpi=600)
\end{lstlisting}
\begin{figure}[H]
\centering
\includegraphics[scale=0.8]{./pic/chapter1/relu_softplus.png}
\end{figure}
\subsection{dropout}
将神经元以概率keep\_prob绝对是否被抑制。如果被抑制该神经元的输出为0如果不被抑制,该神经元的输出将被放大到原来的1/keep\_prop。
默认情况下,每个神经元是否被抑制是相互独立的。但是是否被抑制也可以通过noise\_shape来调节。当noise\_shape[i]=shape(x)[i]时,x中的元素相互独立。如果shape(x)=[k,1,1,n],那么每个批通道都是相互独立的,但是每行每列的数据都是关联的,也就是说要么都为0,要么还是原来的值。
\begin{lstlisting}[language=Python]
import tensorflow as tf
a = tf.constant([[-1.,2.,3.,4.]])
with tf.Session() as sess:
    b = tf.nn.dropout(a,0.5,noise_shape=[1,4])
    print(sess.run(b))
    c = tf.nn.dropout(a,0.5,noise_shape=[1,1])
    print(sess.run(c))
\end{lstlisting}
[[-2.  0.  0.  8.]]\newline
[[-0.  0.  0.  0.]]\newline
当输入数据特征相差明显时,用tanh效果会很好,但在循环过程中会不断扩大特征效果并显示出来。当特征相差不明显时,sigmoid效果比较好。同时,用sigmoid和tanh作为激活函数时,需要对输入进行规范化,否则激活厚的值全部进入平坦区,隐藏层的输出会趋同,丧失原来的特征表达,而relu会好很多,优势可以不需要输入规范化来避免上述情况。因此,现在大部分卷积神经网络都采用relu作为激活函数。
\section{CNN常用函数}
\subsection{卷积函数}
tf.nn.conv2d(input,filter,padding,stride=None,diation\_rate=Nonei每name = None,data\_format=None)\newline
\begin{itemize}
\item input:一个tensor,数据类型必须是float32,或者是float64
\item filter:一个tensor,数据类型必须和input相同。
\item strides:一个长度为4的一组证书类型数组,每一维对应input中每一维对应移动的步数,strides[1]对应input[1]移动的步数。
\item padding:有两个可选参数'VALID'(输入数据维度和输出数据维度不同)和'SAME'(输入数据维度和输出数据维度相同)
\item use\_cudnn\_on\_gpu:一个可选的布尔值,默认情况下时True。
\item name:可选,操作的一个名字。
\end{itemize}
\begin{lstlisting}[language=Python]
import tensorflow as tf
input_data = tf.Variable(tf.random_normal(shape = [10,9,9,3],mean=0,stddev=1),dtype = tf.float32)
kernel = tf.Variable(tf.random_normal(shape = [2,2,3,2],mean = 0,stddev=1,dtype=tf.float32))

y = tf.nn.conv2d(input_data,kernel,strides=[1,1,1,1],padding='SAME')
init = tf.global_variables_initializer()
with tf.Session() as sess:
    sess.run(init)
    print(sess.run(y).shape)
\end{lstlisting}
输出形状为[10,9,9,2]。
\subsection{常见的分类函数}
tf.nn.sigmoid\_cross\_entropy\_with\_logits(logits,targets,name=None)
\begin{itemize}
	\item logits:[batch\_size,num\_classes]
	\item targets:[batch\_size,size]
	\item 输出:loss[batch\_size,num\_classes]
\end{itemize}
最后已成不需要进行sigmoid操作。\par
tf.nn.softmax(logits,dim=-1,name=None):计算Softmax
\[softmax = \frac{x^{logits}}{reduce\_sum(e^{logits},dim)}\]
tf.nn.log\_softmax(logits,dim=-1,name = None)计算log softmax
\[logsoftmax = logits-log(reduce\_softmax(exp(logits),dim))\]
tf.nn.softmax\_cross\_entropy\_with\_logits(\_setinel=None,labels=None,logits=None,dim=-1,name=None)
输出loss:[batch\_size]保存的时batch中每个样本的交叉熵。
tf.nn.sparse\_softmax\_cross\_entropy\_with\_logic(logits,labels,name=None)
\begin{itemize}
	\item logits:神经网络最后一层的结果。
	\item 输入logits:[batch\_size,num\_classes],labels:[batch\_size],必须在[0,num\_classes]
	\item loss[batch],保存的是batch每个样本的交叉熵。
\end{itemize}
\section{优化方法}
\begin{itemize}
	\item tf.train.GradientDescentOptimizer
	\item tf.train.AdadeltaOptimizer
	\item tf.train.AdagradDAOptimizer
	\item tf.train.AdagradOptimizer
	\item tf.train.MomentumOptimizer
	\item tf.train.AdamOptimizer
	\item tf.train.FtrlOptimizer
	\item tf.train.RMSPropOptimizer
\end{itemize}
\subsection{BGD}
BGD(batch gradient descent)批量梯度下降。这种方法是利用现有的参数对训练集中的每一个输入生成一个估计输出$y_i$,然后跟实际的输出$y_i$比较,统计所有的误差,求平均后的到平均误差作为更新参数的依据。啊它的迭代过程是:
\begin{enumerate}
	\item 提取训练集集中所有内容$\{x_1,\ldots,x_n\}$,以及相关的输出$y_i$;
	\item 计算梯度和误差并更新参数。
\end{enumerate}
这种方法的优点是:使用所有数据计算,都保证收敛,并且不需要减少学习率。缺点是每一步需要使用所有的训练数据,随着训练的进行,速度会变慢。那么如果将训练数据拆分成一个个batch,每次抽取一个batch数据更新参数,是不是能加速训练?这就是SGD。
\subsection{SGD}
SGD(stochastic gradient descent):随机梯度下降。这种方法的主要思想是将数据集拆分成一个个的batch,随机抽取一个batch计算并更新参数,所以也称为MBGD(minibatch gradient descent)\
SGD在每次迭代计算mini-batch的梯度,然后对参数进行更新。和BGD相比,SGD在训练数据集很大时也能以较快的速度收敛,但是它有两个缺点:
\begin{enumerate}
\item 需要手动调整学习率,此外选择合适的学习率比较困难。尤其在训练时,我们常常想对常出现的特征更快速的更新,对不常出现的特征更新速度慢些,而SGD更新参数时对所有参数采用一样的学习率,因此无法满足要求。
\item SGD:容易收敛到局部最优。
\end{enumerate}
\subsection{momentum}
Momentum是模拟物理学中的动量概念,更新时在一定程度上保留之前的更新方向,利用当前批次再次微调本次更新参数,因此引入了一个新的变量v,作为前几次梯度的累加。因此,momentum能够更新学习率,在下降初期,前后梯度方向一致时能加速学习:在下降的中后期,在局部最小值附近来回振荡,能够抑制振荡加快收敛。
\subsection{Nesterov Momentum}
标准的Monentum法首先计算一个梯度,然后在加速更新梯度的方向进行一个大的跳跃Nesterov首先在原来加速的梯度方向进行一个大的跳跃,然后在改为值设置计算梯度值,然后用这个梯度值修正最终的更新方向。
\subsection{Adagrad}
Adagrade能够自适应的为各个参数分配不同的学习率,能够控制每个维度的梯度方向,这种方法的优点是能实现学习率的自动更改,如果本次更新时梯度大,学习率就衰减得快,如果这次更新时梯度小,学习率衰减得就慢些。
\subsection{RMSprop}
和Momentum类似,通过引入衰减系数使得每个回合都衰减一定比例。在实践中,对循环神经网络效果很好。
\subsection{Adam}
名称来自自适应矩阵(adaptive moment estimation).Adam根据损失函数针对每个参数的一阶矩,二阶矩估计动态调整每个参数的学习率。
\begin{lstlisting}[language=Python]
import numpy as np
import tensorflow as tf
import matplotlib.pyplot as plt
tf.set_random_seed(0)
np.random.seed(0)
LR = 0.01
BATCH_SIZE = 32
x = np.linspace(-1,1,100).reshape(-1,1)
noise = np.random.normal(0,0.1,size=x.shape)
y = np.power(x,2)+noise
class Net:
    def __init__(self,opt,**kwargs):
        self.x = tf.placeholder(tf.float32,[None,1])
        self.y = tf.placeholder(tf.float32,[None,1])
        l = tf.layers.dense(self.x,20,tf.nn.relu)
        out = tf.layers.dense(l,1)
        self.loss = tf.losses.mean_squared_error(self.y,out)
        self.train = opt(LR,**kwargs).minimize(self.loss)
net_SGD = Net(tf.train.GradientDescentOptimizer)
net_momentum = Net(tf.train.MomentumOptimizer,momentum=0.9)
net_RMSprop = Net(tf.train.RMSPropOptimizer)
net_Adam = Net(tf.train.AdamOptimizer)
nets = [net_SGD,net_momentum,net_RMSprop,net_Adam]
sess = tf.Session()
sess.run(tf.global_variables_initializer())
losses_his = [[],[],[]]
for step in range(300):
    index = np.random.randint(0,x.shape[0],BATCH_SIZE)
    b_x = x[index]
    b_y = y[index]
    for net,l_his in zip(nets,losses_his):
        _,l = sess.run([net.train,net.loss],{net.x:b_x,net.y:b_y})
        l_his.append(l)
labels = ['SGD','Momentum','RMSprop','Adam']
for i,l_his in enumerate(losses_his):
    plt.plot(l_his,label=labels[i])
plt.legend(loc='best')
plt.xlabel('Step')
plt.ylabel('Loss')
plt.ylim(0,0.2)
plt.savefig('Opt.png',dpi=600)
\end{lstlisting}
\begin{figure}[H]
	\centering
	\includegraphics[scale=0.4]{./pic/chapter1/Opt.png}
\end{figure}
\subsection{构造简单的神经网络拟合数据}
原始数据为$y=x^2$的基础上添加随机噪声。原始数据的散点图如下
\begin{center}
\begin{figure}[H]
\includegraphics[scale=0.6]{./pic/chapter1/origin.png}
\end{figure}
\end{center}
\begin{lstlisting}[language=Python]
#tensorflow 1.2.1
import tensorflow as tf
import matplotlib.pyplot as plt
import numpy as np
tf.set_random_seed(0)
np.random.seed(0)
#生成数据
step = 100
x = np.linspace(-1,1,step).reshape(-1,1)
noise = np.random.normal(0,0.1,size=x.shape)
y = np.power(x,2)+noise

tf_x = tf.placeholder(tf.float32,x.shape)
tf_y = tf.placeholder(tf.float32,x.shape)
l1 =  tf.layers.dense(tf_x,10,tf.nn.relu)
output = tf.layers.dense(l1,1)

loss = tf.losses.mean_squared_error(tf_y,output)
optimizer = tf.train.GradientDescentOptimizer(learning_rate=0.5)
train_op = optimizer.minimize(loss)

sess = tf.Session()
sess.run(tf.global_variables_initializer())
plt.ion()
for step in range(100):
    _,l,pred = sess.run([train_op,loss,output],{tf_x:x,tf_y:y})
    if step%5==0:
        plt.cla()
        plt.scatter(x,y)
        plt.title(r'$y=x^2+noise$')
        plt.plot(x,pred,'r-',lw=2)
        plt.text(0,0.8,'Loss=%.4f'%l,fontdict={'size':10,'color':'blue'})
        plt.xlabel("x")
        plt.ylabel(r"$y=x^2$")
        plt.pause(0.1)
plt.ioff()
plt.show()
\end{lstlisting}
最终拟合数据:
\begin{figure}[H]
\includegraphics[scale=0.4]{./pic/chapter1/final.png}
\end{figure}
\section{TensorBoard}
\begin{lstlisting}[language=Python]
import tensorflow as tf
import matplotlib.pyplot as plt

tf.set_random_seed(1)
x0 = tf.random_normal((100,2),2,2,tf.float32,0)
y0 = tf.zeros(100)
x1 = tf.random_normal((100,2),-2,2,tf.float32,0)
y1 = tf.ones(100)
x = tf.reshape(tf.stack((x0,x1),axis=1),(200,2))
y = tf.reshape(tf.stack((y0,y1),axis=1),(200,1))
with tf.Session() as sess:
    x = sess.run(x)
    y = sess.run(y)

tf_x = tf.placeholder(tf.float32, x.shape)     # input x
tf_y = tf.placeholder(tf.int32, y.shape)     # input y

# neural network layers
l1 = tf.layers.dense(tf_x, 10, tf.nn.relu)          # hidden layer
output = tf.layers.dense(l1, 2)                     # output layer

loss = tf.losses.sparse_softmax_cross_entropy(labels=tf_y, logits=output)           # compute cost
accuracy = tf.metrics.accuracy(          # return (acc, update_op), and create 2 local variables
            labels=tf.squeeze(tf_y), predictions=tf.argmax(output, axis=1),)[1]
optimizer = tf.train.GradientDescentOptimizer(learning_rate=0.05)
train_op = optimizer.minimize(loss)

sess = tf.Session()                                                                 # control training and others
init_op = tf.group(tf.global_variables_initializer(), tf.local_variables_initializer())
sess.run(init_op)     # initialize var in graph

plt.ion()   # something about plotting
for step in range(100):
    _, acc, pred = sess.run([train_op, accuracy, output], {tf_x: x, tf_y: y})
    if step % 2 == 0:
        plt.cla()
        plt.scatter(x[:, 0], x[:, 1], c=pred.argmax(1), s=100, lw=0, cmap='RdYlGn')
        plt.text(1.5, -4, 'Accuracy=%.2f' % acc, fontdict={'size': 20, 'color': 'red'})
        plt.pause(0.1)
plt.ioff()
plt.show()
\end{lstlisting}
\begin{figure}[H]
	\includegraphics[scale=0.4]{./pic/chapter1/tenbor1.png}
\end{figure}

