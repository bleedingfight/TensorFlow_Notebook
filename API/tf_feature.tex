\section{tf.feature\_cloumn}
接收表示特征的工具。
\subsection{bucketized\_column}
\begin{lstlisting}[language=Python]
bucketized_column(
    source_column,
    boundaries
)
\end{lstlisting}
代表离散化的稠密输入,bucket包括左边界不包括右边界,也就是说boundaries=[0.,1.,2.]生成buckets(-ing,0.),[0.,1.),[1.,2.),[2,inf)。例如如果输入是:
\begin{lstlisting}[language=Python]
boundaries = [0, 10, 100]
input tensor = [[-5, 10000]
                [150,   10]
                [5,    100]]
\end{lstlisting}
输出是:
\begin{lstlisting}[language=Python]
output = [[0, 3]
          [3, 2]
          [1, 3]]
\end{lstlisting}
例如:\par
\begin{lstlisting}[language=Python]
price = numeric_column('price')
bucketized_price = bucketized_column(price, boundaries=[...])
columns = [bucketized_price, ...]
features = tf.parse_example(..., features=make_parse_example_spec(columns))
linear_prediction = linear_model(features, columns)

# or
columns = [bucketized_price, ...]
features = tf.parse_example(..., features=make_parse_example_spec(columns))
dense_tensor = input_layer(features, columns)
\end{lstlisting}
bucketized\_column也能被用crossed\_column传递:
\begin{lstlisting}[language=Python]
price = numeric_column('price')
# bucketized_column converts numerical feature to a categorical one.
bucketized_price = bucketized_column(price, boundaries=[...])
# 'keywords' is a string feature.
price_x_keywords = crossed_column([bucketized_price, 'keywords'], 50K)
columns = [price_x_keywords, ...]
features = tf.parse_example(..., features=make_parse_example_spec(columns))
linear_prediction = linear_model(features, columns)
\end{lstlisting}
参数:
\begin{itemize}
	\item source\_column:numeric\_column生成的一维稠密列
	\item boundaries:指定边界的一个浮点数列表或元组
	\item[Returns]:一个\_BucketizedColumn
	\item[Raises]:
	\begin{itemize}
		\item ValueError:如果source\_column不是一个数值列,或者不是一维。
		\item ValueError:如果boundaries不是一个列表或者元组。
	\end{itemize}
\end{itemize}
\subsection{categorical\_column\_with\_hash\_bucket}
\begin{lstlisting}[language=Python]
categorical_column_with_hash_bucket(
    key,
    hash_bucket_size,
    dtype=tf.string
)
\end{lstlisting}
当你的稀疏特征是字符串或者整数格式的时候你想布置你的输入进入一个有限的散列的bucket,output\_id=Hash(input\_feature\_string)\%bucket\_size输入不字典features,features[key]可以是Tensor或者SparseTensor。如果是Tensor整数缺少值用-1表示,字符串用''表示。注意这些值独立于参数default\_value参数.
\begin{lstlisting}[language=Python]
keywords = categorical_column_with_hash_bucket("keywords", 10K)
columns = [keywords, ...]
features = tf.parse_example(..., features=make_parse_example_spec(columns))
linear_prediction = linear_model(features, columns)

# or
keywords_embedded = embedding_column(keywords, 16)
columns = [keywords_embedded, ...]
features = tf.parse_example(..., features=make_parse_example_spec(columns))
dense_tensor = input_layer(features, columns)
\end{lstlisting}
参数:
\begin{itemize}
	\item key:一个用来识别输入的独一无二的字符串。它用做列的名字和特征的key。
	\item hash\_bucket\_size:一个大于1的整数,bucket的树木。
	\item dtype:feature的类型。仅仅字符串和整数型支持、
	\item[Returns] 一个\_HashedCategoricalColumn
	\item[Raises]:
	\begin{itemize}
		\item ValueError:hash\_bucket\_size不大于1。
		\item ValueError:dtype不是字符串或者整数。
	\end{itemize}
\end{itemize}
\subsection{categorical\_column\_with\_identity}
\begin{lstlisting}[language=Python]
categorical_column_with_identity(
    key,
    num_buckets,
    default_value=None
)
\end{lstlisting}
一个返回identity值的\_categoricalColumn,当你的输入是[0,num\_buckets)之间的整数,你想用哪个输入作为分类的ID,值在范围外的为默认值是指定,否则将不能指定。通常被用于整数索引的连续范围,但是它不是必须的如归哦一些ID没有使用这就会很低效考虑使用categorical\_column\_with\_hash\_bucket,对于输入字典features,features[key]可以是Tensor或者SparseTensor,如果是Tensor,用-1表示整数类型的缺失,''表示字符串类型的缺失,注意值独立于default\_value参数,下面的例子中每个输入在[0,100000)中每指定相同的值,另一个输入赋值default\_value 0。

线性模型:
\begin{lstlisting}[language=Python]
video_id = categorical_column_with_identity(
    key='video_id', num_buckets=1000000, default_value=0)
columns = [video_id, ...]
features = tf.parse_example(..., features=make_parse_example_spec(columns))
linear_prediction, _, _ = linear_model(features, columns)
\end{lstlisting}
迁入DNN模型:
\begin{lstlisting}[language=Python]
columns = [embedding_column(video_id, 9),...]
features = tf.parse_example(..., features=make_parse_example_spec(columns))
dense_tensor = input_layer(features, columns)
\end{lstlisting}
参数:
\begin{itemize}
	\item 一个独一无二的字符串识别输入特征。被用作特征解析配置列表名字和字典的键,特征对象和特征列。
	\item num\_bucket范围为输入和输出[0,num\_buckets)
	\item default\_value:如果为None,列的图操作将因为草畜输入范围而失败,因此它的值必须在[0,num\_buckets)
	\item[Returns]:一个返回识别值的\_CategoricalColumn
	\item[Raises]:
	\begin{itemize}
		\item ValueError:如果num\_buckets小于1。
		\item ValueError:如果default\_value不在[0,num\_buckets)
	\end{itemize}
\end{itemize}
\subsection{categorical\_column\_with\_vocabulary\_file}
\begin{lstlisting}[language=Python]
categorical_column_with_vocabulary_file(
    key,
    vocabulary_file,
    vocabulary_size,
    num_oov_buckets=0,
    default_value=None,
    dtype=tf.string
)
\end{lstlisting}
一个包含吃会文件的\_CategoricalColumn,当你的输入是字符串或者整数格式时使用,你有一个词汇文件映射每个值到一个整数ID,模型,草果范围的值被忽略,num\_oov\_buckets和default\_value指定如何包含超出词汇表范围的值,对于输入字典features,features[key]可以是Tensor或者SparseTensor,如果是Tensor,整数型缺少值-1表示,字符串缺少值用''表示,注意这些值是独立于参数default\_value参数,样本num\_oov\_buckets:文件'/us/states.txt'包含50行每行美国州的所写的两个字符,所有文件中的输入值被负值为ID0-49,对应的行数,所有其他得知被打散赋值ID 50-54。
\begin{lstlisting}[language=Python]
states = categorical_column_with_vocabulary_file(
    key='states', vocabulary_file='/us/states.txt', vocabulary_size=50,
    num_oov_buckets=5)
columns = [states, ...]
features = tf.parse_example(..., features=make_parse_example_spec(columns))
linear_prediction = linear_model(features, columns)
\end{lstlisting}
例如default\_value:文件'us/state.txt'包含51行,第一行是'XX',其它的50行有
两个美国州的缩写,'XX'和其他文件中缺失的值被赋值为ID0,所有其它值在1-50。
\begin{lstlisting}[language=Python]
states = categorical_column_with_vocabulary_file(
    key='states', vocabulary_file='/us/states.txt', vocabulary_size=51,
    default_value=0)
columns = [states, ...]
features = tf.parse_example(..., features=make_parse_example_spec(columns))
linear_prediction, _, _ = linear_model(features, columns)
\end{lstlisting}
增加一个embedding:
\begin{lstlisting}[language=Python]
columns = [embedding_column(states, 3),...]
features = tf.parse_example(..., features=make_parse_example_spec(columns))
dense_tensor = input_layer(features, columns)
\end{lstlisting}
参数:
\begin{itemize}
	\item key:一个独一无二的识别输入特征的字符串,他被用作列的名字。字典的用于特征解析配置,特征Tensor对象和其它列的键。
	\item vocabulary\_file:词汇表的文件名。
	\item vocabulary\_size:词汇表中的元素的个数,必须不大于vocabulary\_file的长度,如果小于长度之后的值将被忽略。
	\item num\_oov\_buckets:一个非负整数超过词汇bucket的数量,所有超过词汇表的输入将被赋值为在基于散列输入[vocabulary,vocabulary\_size+num\_oov\_buckets)范围的ID,一个正的num\_oov\_buckets不能被default\_value指定。
	\item default\_value:超过特征值范围的整数ID,默认为-1,这可能不被一个正num\_oov\_buckets指定。
	\item dtype:特征的类型,仅仅字符串和整数支持
	\item[Returns]一个词汇表文件\_CategoricalColumn
	\item[Raise]:ValueError:Vacabulary\_file文件缺失,vocabulary\_size缺失或者小于1,num\_oov\_buckets是一个负整数,num\_oov\_buckets和default\_value被同时指定,dtype不是整数或者字符串。
\end{itemize}
\subsection{categorical\_column\_with\_vocabulary\_list}
\begin{lstlisting}[language=Python]
categorical_column_with_vocabulary_list(
    key,
    vocabulary_list,
    dtype=None,
    default_value=-1,
    num_oov_buckets=0
)
\end{lstlisting}
一个存储词汇表\_CategoricalColumn,当你的输入是一个字符串或者整数格式时使用,你有一个存储字徽标映射,每个值到一个整数ID、,默认超过词汇表范围的值被忽略,用num\_oov\_buckets和default\_value指定如何包含操作词汇表的值,对于字典features,features[key]既可以是Tensor也可以是SparseTensor,如果是Tensor,整数型缺失值用-1表示,字符串缺失值用''表示注意这些值独立于default\_value参数,num\_oov\_buckets样本:在下面的例子中,每个vocabulary\_list中的输入被赋值ID0-3对应他的索引,所有的其他输入被打散赋值为4-5.
\begin{lstlisting}[language=Python]
colors = categorical_column_with_vocabulary_list(
    key='colors', vocabulary_list=('R', 'G', 'B', 'Y'),
    num_oov_buckets=2)
columns = [colors, ...]
features = tf.parse_example(..., features=make_parse_example_spec(columns))
linear_prediction, _, _ = linear_model(features, columns)
\end{lstlisting}
样本default\_value:下面的例子每个vocabulary\_list的输入被赋值一个ID0-4对应他的索引,所有其它的输入被复制defaulr\_vale 0.
\begin{lstlisting}[language=Python]
colors = categorical_column_with_vocabulary_list(
    key='colors', vocabulary_list=('X', 'R', 'G', 'B', 'Y'), default_value=0)
columns = [colors, ...]
features = tf.parse_example(..., features=make_parse_example_spec(columns))
linear_prediction, _, _ = linear_model(features, columns)
\end{lstlisting}
做一个embedding:
\begin{lstlisting}[language=Python]
columns = [embedding_column(colors, 3),...]
features = tf.parse_example(..., features=make_parse_example_spec(columns))
dense_tensor = input_layer(features, columns)
\end{lstlisting}
参数:
\begin{itemize}
	\item key:一个独一无二的字符串识别输入特征。它被用作列的名字和特征解析配置的字典的key,特征Tensor对象和特征列。
	\item vocabulary\_list:一个定义可顺序列带的词汇表,每个特征被映射到它在vocabulary\_list的值的索引必须是可转换的类型.
	\item dtype:特征的类型,仅仅支持字符串和整数类型。如果为None它将从vocabulary\_list推理。
	\item default\_value:超过词汇表返回的整数ID值,默认为-1,不能被指定为一个正的num\_oov\_buckets.
	\item num\_oov\_buckets:非负整数,超过词汇表bucket的数量,所有的处处词汇表的输入将被赋值ID[len(vocabulary\_list),len(vocabulary\_list)+num\_oov\_buckets)基于山裂输入值,一个正的num\_oov\_buckets不能被default\_value指定。
	\item[Returns]:一个存储的词汇表\_CategoricalColumn。
	\item[Raise]:ValueError:如果vocabulary\_list,num\_oov\_buckets是负整数,num\_oov\_buckets和default\_value同时被指定,dtype不是整数或者字符串。
\end{itemize}
\subsection{cross\_column}
\begin{lstlisting}[language=Python]
crossed_column(
    keys,
    hash_bucket_size,
    hash_key=None
)
\end{lstlisting}
返回执行交叉分类特征的一列,交叉特征被hash\_bucket\_size打散,变换可能被Hash\%hash\_bucket\_size打散,例如输入特征是:
\begin{itemize}
	\item 第一个键值访问的SparseTensor。
	\begin{lstlisting}[language=Python]
	shape = [2, 2]
{
    [0, 0]: "a"
    [1, 0]: "b"
    [1, 1]: "c"
}
	\end{itemize}
	\item被第二个键值访问的SparseTensor:
	\begin{itemize}
	shape = [2, 1]
{
    [0, 0]: "d"
    [1, 0]: "e"
}
	\end{lstlisting}
\end{itemize}
交叉特征:
\begin{lstlisting}[language=Python]
 shape = [2, 2]
{
    [0, 0]: Hash64("d", Hash64("a")) % hash_bucket_size
    [1, 0]: Hash64("e", Hash64("b")) % hash_bucket_size
    [1, 1]: Hash64("e", Hash64("c")) % hash_bucket_size
}
\end{lstlisting}
通过交叉字符串特征创建一个线性模型:
\begin{lstlisting}[language=Python]
keywords = categorical_column_with_vocabulary_file(
    'keywords', '/path/to/vocabulary/file', vocabulary_size=1K)
keywords_x_doc_terms = crossed_column([keywords, 'doc_terms'], 50K)
columns = [keywords_x_doc_terms, ...]
features = tf.parse_example(..., features=make_parse_example_spec(columns))
linear_prediction = linear_model(features, columns)
\end{lstlisting}
如果输入特征是数值类型,你可以用categoriacal\_column\_with\_identity或者bucketized\_column:
\begin{lstlisting}[language=Python]
# vertical_id is an integer categorical feature.
vertical_id = categorical_column_with_identity('vertical_id', 10K)
price = numeric_column('price')
# bucketized_column converts numerical feature to a categorical one.
bucketized_price = bucketized_column(price, boundaries=[...])
vertical_id_x_price = crossed_column([vertical_id, bucketized_price], 50K)
columns = [vertical_id_x_price, ...]
features = tf.parse_example(..., features=make_parse_example_spec(columns))
linear_prediction = linear_model(features, columns)
\end{lstlisting}
为了在DNN模型中使用交叉列,你需要增加它到一个嵌入列:
\begin{lstlisting}[language=Python]
vertical_id_x_price = crossed_column([vertical_id, bucketized_price], 50K)
vertical_id_x_price_embedded = embedding_column(vertical_id_x_price, 10)
dense_tensor = input_layer(features, [vertical_id_x_price_embedded, ...])
\end{lstlisting}
参数:
\begin{itemize}
	\item keys:一个可以迭代的识别被交叉的特征,每个元素可以是字符串(用相关的特征必须是字符串类型),\_CategoricalColumn:将用变形的列通过tensor生成,不支持散列的绝对的列。
	\item hash\_bucket\_size:一个大于1的整数,buckets的数量。
	\item hash\_key:指定将被FingerprintCat64使用的hash\_key结合SparseCrossOp fingerprint交叉。
	\item[Returns]:一个\_CrossedColumn
	\item[Raises]:ValueError:如果len(keys)<2,如果任何keys既不是字符串也不是\_CategoricalColumn,如果任何keys是\_ashedCategoricalColumn,如果hash\_buckets\_size<1
\end{itemize}
\subsection{embedding\_column}
\begin{lstlisting}[language=Python]
embedding_column(
    categorical_column,
    dimension,
    combiner='mean',
    initializer=None,
    ckpt_to_load_from=None,
    tensor_name_in_ckpt=None,
    max_norm=None,
    trainable=True
)
\end{lstlisting}
从稀疏绝对输转换\_CategoricalColumn,当你的输入稀疏但是你想转换他们为一个稠密的表达时使用,输入必须被任何 categorical\_column\_*函数创建的一个\_CategoricalColumn,下面的雷子是一个识别DNN模型列的例子:
\begin{lstlisting}[language=Python]
video_id = categorical_column_with_identity(
    key='video_id', num_buckets=1000000, default_value=0)
columns = [embedding_column(video_id, 9),...]
features = tf.parse_example(..., features=make_parse_example_spec(columns))
dense_tensor = input_layer(features, columns)
\end{lstlisting}
参数:
\begin{lstlisting}[language=Python]
\item categorical\_column:一个categorical\_column\_with\_*函数创建的\_CategoricalColumn,这个列生成代表嵌入的查找的稀疏ID。
\item dimension:一个整数指定嵌入的维度,必须大于0.
\item combiner:一个字符串指定如果有多个输入在一行时如何减少,当前的'mean','sqrtn'和'sum'被支持,'mean'默认,'sqrtn'经常得到好的精度,特别是词袋模型,每一个可以被认为每列作为样本级别的正规化。对于更多信息,查看tf.embedding\_lookup\_sparse.
\item initializer:一个变量初始化器函数用于embedding变量的初始化,如果没有在一定默认为tf.truncated\_normal\_initializer,均值为0标准差为1/sqrt(demension).
\item ckpt\_to\_load\_from:string代表从保存列的权重checkpoint名字、样例,如果tensor\_name\_load\_from不是None
\item max\_norm:如果不为None,嵌入值用l2中正规化
\iten trainable:驶入嵌入能被训练,默认为true
\item[Returns]:从稀疏输入转化的\_DenseColumn
\item [Raises]:ValueError,如果dimension不大于0,如果一个chpt\_to\_load\_from和tensor\_name\_in\_ckpt被指定,如果initializer被指定不被调用。
\end{lstlisting}
\subsection{indicator\_column}
代表给定绝对的mult-hot,用于打包任何categorical\_column\_* 用embedding\_column如果输入是稀疏的:
\begin{lstlisting}[language=Python]
name = indicator_column(categorical_column_with_vocabulary_list(
    'name', ['bob', 'george', 'wanda'])
columns = [name, ...]
features = tf.parse_example(..., features=make_parse_example_spec(columns))
dense_tensor = input_layer(features, columns)

dense_tensor == [[1, 0, 0]]  # If "name" bytes_list is ["bob"]
dense_tensor == [[1, 0, 1]]  # If "name" bytes_list is ["bob", "wanda"]
dense_tensor == [[2, 0, 0]]  # If "name" bytes_list is ["bob", "bob"]
\end{lstlisting}
参数:
categorical\_column:一个由categorical\_column\_with*或者crossed\_column创建的\_CategoricalColumn,返回一个\_IndicatorColumn。
\subsection{input\_layer}
\begin{lstlisting}[language=Python]
input_layer(
    features,
    feature_columns,
    weight_collections=None,
    trainable=True
)
\end{lstlisting}
基于给定的feature\_column返回一个出米的Tensor作为输入层,通常在训练数据上被FeatureColumns描述的见得样本,在模型的第一层列的方向数据应该被转化为一个Tensor。
例如:
\begin{lstlisting}[language=Python]
price = numeric_column('price')
keywords_embedded = embedding_column(
    categorical_column_with_hash_bucket("keywords", 10K), dimensions=16)
columns = [price, keywords_embedded, ...]
features = tf.parse_example(..., features=make_parse_example_spec(columns))
dense_tensor = input_layer(features, columns)
for units in [128, 64, 32]:
  dense_tensor = tf.layers.dense(dense_tensor, units, tf.nn.relu)
prediction = tf.layers.dense(dense_tensor, 1)
\end{lstlisting}
参数:
\begin{itemize}
	\item features:一个从key映射到tensor。\_FeatureColumn通过通过这些key查找,例如numeric\_column('price')将查看字典中的'price',值可以是SparseTensor护着Tensor取决于\_FeatureColumn。
	\item feature\_columns:一个可迭代的包含FeatureColumn用作你的模型的输入所有的这些都应该是继承自\_DenseColumn的实例,像numeric\_column,embedding\_column,bucketized\_column,indicator\_column,如果你有绝对的特征,你可以用embedding\_column或者indicator\_column打包他们。
	\item weight\_collections:一个将被添加到集合列表的名字,注意变量也被增加到集合tf.GraphKeys.GLOBAL\_VASRIABLE和ops.GraphKeys.MODEL\_VARIABLES.
	\item trainable:如果为True添加变量到图集合GraphKeys.TRAINABLE\_VARIABLE.
	\item[Returns]:一个代表模型输入层Tensor,他的形状是(batch\_size,first\_layer\_dimension)数据类型是float32,first\_layer\_demension倍feature\_columns决定。
	\item[Raises]:如果一个item在feature\_columns不是一个\_DenseColumn。
\end{itemize}
\subsection{linear\_model}
\begin{lstlisting}[language=Python]
linear_model(
    features,
    feature_columns,
    units=1,
    sparse_combiner='sum',
    weight_collections=None,
    trainable=True
)
\end{lstlisting}
返回一个基于feature\_columns给定的线性预测Tensor,返回一个基于给定feature\_column线性预测,这个函数生成基于输出维度单位的权重和,权重和在分类问题上涉及分类问题。它涉及线性回归模型问题的预测,注意不支持列:linear\_model对待绝对列作为indicator\_column当input\_layer明确要求用一个embedding\_column或者indicator\-column打包他们。
例如:
\begin{lstlisting}[language=Python]
price = numeric_column('price')
price_buckets = bucketized_column(price, boundaries=[0., 10., 100., 1000.])
keywords = categorical_column_with_hash_bucket("keywords", 10K)
keywords_price = crossed_column('keywords', price_buckets, ...)
columns = [price_buckets, keywords, keywords_price ...]
features = tf.parse_example(..., features=make_parse_example_spec(columns))
prediction = linear_model(features, columns)
\end{lstlisting}
参数:
\begin{itemize}
	\item features:一个从key到tensor的映射,\_FeatureColumn查找这些键,例如numeric\_column('price')将查找字典中的'price',Value是Tensor或者SparseTensor取决于\_FeatureColumn.
	\item feature\_columns:一个可迭代的FeatureColumn用做模型的输入,所有的项目都应该是继承自\_FeatureColumn的实体。
	\item units:一个整数,输出空间的维度,默认值为1。
	\item sparse\_combiner:一个字符串指定如果稀疏列是多值时如何减少的字符串,当前"mean","sqrtn"和"sum支持,sum默认是sqrtn经常获得好的精度。特别的在字列的词袋模型,它结合每个列:
	\begin{itemize}
	\item sum:没有正规化列中的特征
	\item mean:在列的特征上做l1正则化
	\item sqrtn:在猎德特征上做l2正则化。
	\end{itemize}
	\item weight\_collections:一个每个变量将被添加到集合的名字,注意变量将被添加到集合tf.GraphKeys.GLOBAL\_VARIABLE和ops.Graphkeys.MODEL\_VARIABLE
	\item trainable:如果为Treu将增加变量到图集合Graphkeys.TRAINABLE\_VARIABLE
	\item[Returns]:一个代表预测线性模型的预测的Tensor,它的形状(batch\_size,units)和它的dtype为float32
	\item[Raises]:ValueError,如果在feature\_columns中的项目既不是一个\_DenseColumn也不是\_CategoricalColumn
\end{itemize}
\subsection{make\_sparse\_example\_spec}
从输入特征列创建稀疏空间字典,返回字典可以被用作tf.parse\_example的features参数。
\begin{lstlisting}[language=Python]
# Define features and transformations
feature_b = numeric_column(...)
feature_c_bucketized = bucketized_column(numeric_column("feature_c"), ...)
feature_a_x_feature_c = crossed_column(
    columns=["feature_a", feature_c_bucketized], ...)

feature_columns = set(
    [feature_b, feature_c_bucketized, feature_a_x_feature_c])
features = tf.parse_example(
    serialized=serialized_examples,
    features=make_parse_example_spec(feature_columns))
\end{lstlisting}
下面的例子,make\_sparse\_example\_spec将返回字典:
\begin{lstlisting}[language=Python]
{
    "feature_a": parsing_ops.VarLenFeature(tf.string),
    "feature_b": parsing_ops.FixedLenFeature([1], dtype=tf.float32),
    "feature_c": parsing_ops.FixedLenFeature([1], dtype=tf.float32)
}
\end{lstlisting}
参数:feature\_columns:一个包含所有特征列的迭代器,所有的项目应该是继承自\_FeatureColumn的实体。返回一个映射每个特征键到FixedlenFeature或者VarLenFeature值,异常ValueError:如果任何给定的feature\_columns不包含一个\_FeatureColumn实体。
\subsection{numeric\_column}
\begin{lstlisting}[language=Python]
numeric_column(
    key,
    shape=(1,),
    default_value=None,
    dtype=tf.float32,
    normalizer_fn=None
)
\end{lstlisting}
表示真实值或者数值特征,例如:
\begin{lstlisting}[language=Python]
price = numeric_column('price')
columns = [price, ...]
features = tf.parse_example(..., features=make_parse_example_spec(columns))
dense_tensor = input_layer(features, columns)

# or
bucketized_price = bucketized_column(price, boundaries=[...])
columns = [bucketized_price, ...]
features = tf.parse_example(..., features=make_parse_example_spec(columns))
linear_prediction = linear_model(features, columns)
\end{lstlisting}
参数:
\begin{itemize}
	\item key:一个独一无二的字符串识别输入特征,它被用作特征解析配置,特征Tensor对象和特征列的名字
	\item shape:一个指定Tensor形状的迭代器整数,一个整数可以给单一为,一个整数意味着一个Tensor的宽,Tensor代表将有形状[batch\_size]+shape
	\item dtype:定义值的数据类型,默认是tf.float32,必须是一个没有量化的,真实的整数或者浮点数类型。
	\item normalizer\_fn:如果不为None,一个函数可以在default\_value被用于解析后背正则化值。睁着花旗函数接受输入Tensor作为参数返回输出Tensor,请注意尽管多数情况用这个函数证这话,他可以没用于多种Tensorflow转换。
	\item[Return] \_NumbericColumn
	\item[Raise]:如果任何维度不是证书有,default\_value是一个迭代不但是形状不匹配,default\_value不兼容的数据类型报TypeError,如果任何维度不是整数或者不能转化为tf.float32的dtype将报ValueError。
\end{itemize}
\subsection{weighted\_catrgorical\_column}
\begin{lstlisting}[language=Python]
weighted_categorical_column(
    categorical_column,
    weight_feature_key,
    dtype=tf.float32
)
\end{lstlisting}
应用权重到\_CategoricalColumn,让输入是一维有值时使用,例如如果你表示文本文件作为一个子频率的集合,你可以提供两个平行的稀疏输入特征,输入tf.Example对象:
\begin{lstlisting}[language=Python]
[
  features {
    feature {
      key: "terms"
      value {bytes_list {value: "very" value: "model"}}
    }
    feature {
      key: "frequencies"
      value {float_list {value: 0.3 value: 0.1}}
    }
  },
  features {
    feature {
      key: "terms"
      value {bytes_list {value: "when" value: "course" value: "human"}}
    }
    feature {
      key: "frequencies"
      value {float_list {value: 0.4 value: 0.1 value: 0.2}}
    }
  }
]
\end{lstlisting}
\begin{lstlisting}[language=Python]
categorical_column = categorical_column_with_hash_bucket(
    column_name='terms', hash_bucket_size=1000)
weighted_column = weighted_categorical_column(
    categorical_column=categorical_column, weight_feature_key='frequencies')
columns = [weighted_column, ...]
features = tf.parse_example(..., features=make_parse_example_spec(columns))
linear_prediction, _, _ = linear_model(features, columns)
\end{lstlisting}
假设输入字典包含一个key‘terms’的SparseTensor,SparseTensor键'frequencies',两个Tensor必须有相同的索引和dense形状。

参数:
\begin{itemize}
	\item categorical\_column:一个catrgorical\_column\_with\*函数创建的\_CategoricalColumn
	\item weight\_feature\_key:权重值的字符串键。
	\item dtype:权重的类型,像tf.float32,仅仅浮点数和整数支持
	\item[Returns]一个两个稀疏响亮的组合的\_CategoricalColumn:一个代表id另一个代表id特征的权重值。
	\item[Raises]:ValueError:如果dtype不能转化为float32
	\end{itemize}
