\section{tf.contrib.data}
\subsection{Dataset类}
一个Dataset可以被用于代表一个输入pipeline作为元素的集合和一个在这些元素逻辑计划转换。
\begin{tabular}{|c|c|c|}
属性&返回值类型&说明\\
output\_shapes&表示数据集元素组成的迭代结构的tf.TensorShape&数据集组成元素的形状\\
output\_types&迭代结构的tf.DType&返回数据集组成元素的类型&\\
\end{tabular}
方法
\subsubsection{\_\_init\_\_}
\subsubsection{batch}
\subsubsection{chache}
\subsubsection{concatenate}
\subsubsection{dense\_to\_sparse\_batch}
\subsubsection{enumerate}
\subsubsection{filter}
\subsubsection{flat\_map}
\subsubsection{from\_sparse\_tensor\_slices}
\subsubsection{from\_tensors}
\subsubsection{group\_by\_window}
\subsubsection{ignore\_errors}
\subsubsection{interleave}
\subsubsection{list\_files}
\subsubsection{make\_dataset\_resource}
\subsubsection{make\_initializable\_iterator}
\subsubsection{make\_one\_shot\_iterator}
\subsubsection{map}
\subsubsection{padded\_batch}
\subsubsection{range}
\subsubsection{read\_batch\_features}
\subsubsection{repeat}
\subsubsection{shauffle}
\subsubsection{skip}
\subsubsection{take}
\subsubsection{unbatch}
\subsubsection{zip}
