\section{用GPU}
在Tensorflow中CPU,GPU用字符串表示
\begin{itemize}
\item "cpu:0":机器上的CPU
\item "gpu:0":机器上的GPU
\item "gpu:1":机器上的第二块GPU
\end{itemize}
 如果TensorFLow操作有GPU和CPU实现,GPU将被优先指定,例如matmul有CPU和GPU内核,在系统上 有cpu:0和gpu:0,gpu:0将优先运行matmul。
布置采集设备

找到你的操作和tensor上的设备,创建一个会话log\_device\_placement配置设置为True
\begin{python}
import tensorflow as tf
a = tf.reshape(tf.linspace(-1.,1.,12),(3,4))
b = tf.reshape(tf.sin(a),(4,3))
c = tf.matmul(a,b)
with tf.Session() as sess:
    print(sess.run(c))
\end{python}
输出参数:\newline
\includegraphics{./pic/chapter2/use_gpu1.png}\newline
\subsection{手工配置设备}
如果你想将你的操作运行在指定的设备中而不由tensorflow是自动为你选择,你可以用tf.device 创建一个设备,左右的操作将在同一个设备上指定。
\begin{python}
import tensorflow as tf
with tf.device('/cpu:0'):
    a = tf.constant([1.,2.,3.,4.,5.,6.],shape = (2,3),name = 'a')
    b = tf.reshape(a,shape=(3,2))
    c = tf.matmul(a,b)
    with tf.Session(config = tf.ConfigProto(log_device_placement=True)) as sess:
        print(sess.run(c))
\end{python}
\includegraphics[scale=0.4]{use_gpu2.png}\newline
正如你看到的a,b被复制到cpu:0,因为设备没有明确指定,Tensorflow将选择操作和可用的设备(gpu:0)

\subsection{允许GPU的内存增长}

默认情况下Tensorflow将映射所有的CPUs的显存到进程上,用相对精确的GPU内存资源减少内存的碎片化会更高效。 通常有些程序希望分贝可用内存的一部分,或者增加内存的需要两。在会话中tensorflow提供了两个参数 控制它。 第一个参数是allow\_growth选项,根据运行情况分配GPU内存:它开始分配很少的内存,当Session开始运行 需要更多GPU内存是,我们同感Tensorflow程序扩展GPU的内存区域。注意我们不释放内存,因此这可能导致更多的内存碎片。为了开启这个选项,可以通过下面的设置
\begin{python}
config = tf.ConfigProto()
config.gpu_option.allow_growth = True
sess = tf.Session(config=config,...)
\end{python}
第二种方法是per\_process\_gpu\_memory\_fraction选项,决定GPU总体内存中多少应给被分配,例如你可以告诉 Tensorflow分配40\%的GPU总体内存。
\begin{python}
config = tf.ConfigProto()
config.gpu_option.per_process_gpu_memory_fraction = 0.4
sess = tf.Session(config = config)
\end{python}
如果你想限制Tensorflow程序的GPU使用量,这个参数是很有用的。

在多GPU系统是使用GPU

如果你的系统上有超过一个GPU,你的GPU的抵消的ID将被默认选中,如果你想运行在不同的GPU上,你需要指定 你想要执行运算的GPU
\begin{python}
import tensorflow as tf
with tf.device('/gpu2:0'):
    a = tf.constant([1.,2.,3.,4.,5.,6.],shape = (2,3),name = 'a')
    b = tf.reshape(a,shape=(3,2))
    c = tf.matmul(a,b)
    with tf.Session(config = tf.ConfigProto(log_device_placement=True)) as sess:
        print(sess.run(c))
\end{python}
如果你指定的设备不存在,你将个到一个InvalidArgumentError:\newline
\includegraphics[scale=0.4]{use_gpu3.png}\newline
如果你想Tensorflow在万一指定的设备不存在时自动选择一个存在的设备,你可以在创建会话时配置中设置allow\_soft\_placement 为True
\begin{python}
with tf.device('/gpu:2'):
  a = tf.constant([1.,2.,3.,4.,5.,6.],shape= [3,2],name = 'a')
  b = tf.constant([1.,2.,3.,4.,5.,6.],shape= [2,3],name = 'b')
  c = tf.matmul(a,b)
with tf.Session(config = tf.ConfigProto(allow_soft_placement=True,log_device_placement=True)) as sess:
    print(sess.run(c))
\end{python}
\includegraphics[scale=0.4]{use_gpu4.png}\newline
用多GPU

如果你想在多张GPU上运行Tensorflow,你可以在multi-tower fashion上构造你的模型,每个tower 被指定到不同的GPU上。例如:
\begin{python}
c = []
for d in ['/gpu:0', '/gpu:1']:
    with tf.device(d):
        a = tf.constant([1.0, 2.0, 3.0, 4.0, 5.0, 6.0], shape=[2, 3])
        b = tf.constant([1.0, 2.0, 3.0, 4.0, 5.0, 6.0], shape=[3, 2])
        c.append(tf.matmul(a, b))
    with tf.device('/cpu:0'):
        sum = tf.add_n(c)
   # Creates a session with log_device_placement set to True.
sess = tf.Session(config=tf.ConfigProto(allow_soft_placement=True,log_device_placement=True))
  # Runs the op.
print(sess.run(sum))
sess.close()
\end{python}
