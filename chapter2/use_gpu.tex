\section{用GPU}
在Tensorflow中CPU,GPU用字符串表示
\begin{itemize}
\item "cpu:0":机器上的CPU
\item "gpu:0":机器上的GPU
\item "gpu:1":机器上的第二块GPU
\end{itemize}
 如果TensorFLow操作有GPU和CPU实现,GPU将被优先指定,例如matmul有CPU和GPU内核,在系统上 有cpu:0和gpu:0,gpu:0将优先运行matmul。
布置采集设备

找到你的操作和tensor上的设备,创建一个会话log\_device\_placement配置设置为True
\begin{python}
import tensorflow as tf
a = tf.reshape(tf.linspace(-1.,1.,12),(3,4))
b = tf.reshape(tf.sin(a),(4,3))
c = tf.matmul(a,b)
with tf.Session() as sess:
    print(sess.run(c))
\end{python}
输出参数:\newline
\includegraphics{./pic/chapter2/use_gpu1.png}\newline
\subsection{手工配置设备}
如果你想将你的操作运行在指定的设备中而不由tensorflow是自动为你选择,你可以用tf.device 创建一个设备,左右的操作将在同一个设备上指定。
\begin{python}
import tensorflow as tf
with tf.device('/cpu:0'):
    a = tf.constant([1.,2.,3.,4.,5.,6.],shape = (2,3),name = 'a')
    b = tf.reshape(a,shape=(3,2))
    c = tf.matmul(a,b)
    with tf.Session(config = tf.ConfigProto(log_device_placement=True)) as sess:
        print(sess.run(c))
\end{python}
\includegraphics[scale=0.4]{use_gpu2.png}\newline
正如你看到的a,b被复制到cpu:0,因为设备没有明确指定,Tensorflow将选择操作和可用的设备(gpu:0)

\subsection{允许GPU的内存增长}

默认情况下Tensorflow将映射所有的CPUs的显存到进程上,用相对精确的GPU内存资源减少内存的碎片化会更高效。 通常有些程序希望分贝可用内存的一部分,或者增加内存的需要两。在会话中tensorflow提供了两个参数 控制它。 第一个参数是allow\_growth选项,根据运行情况分配GPU内存:它开始分配很少的内存,当Session开始运行 需要更多GPU内存是,我们同感Tensorflow程序扩展GPU的内存区域。注意我们不释放内存,因此这可能导致更多的内存碎片。为了开启这个选项,可以通过下面的设置
\begin{python}
config = tf.ConfigProto()
config.gpu_option.allow_growth = True
sess = tf.Session(config=config,...)
\end{python}
第二种方法是per\_process\_gpu\_memory\_fraction选项,决定GPU总体内存中多少应给被分配,例如你可以告诉 Tensorflow分配40\%的GPU总体内存。
\begin{python}
config = tf.ConfigProto()
config.gpu_option.per_process_gpu_memory_fraction = 0.4
sess = tf.Session(config = config)
\end{python}
如果你想限制Tensorflow程序的GPU使用量,这个参数是很有用的。

在多GPU系统是使用GPU

如果你的系统上有超过一个GPU,你的GPU的抵消的ID将被默认选中,如果你想运行在不同的GPU上,你需要指定 你想要执行运算的GPU
\begin{python}
import tensorflow as tf
with tf.device('/gpu2:0'):
    a = tf.constant([1.,2.,3.,4.,5.,6.],shape = (2,3),name = 'a')
    b = tf.reshape(a,shape=(3,2))
    c = tf.matmul(a,b)
    with tf.Session(config = tf.ConfigProto(log_device_placement=True)) as sess:
        print(sess.run(c))
\end{python}
如果你指定的设备不存在,你将个到一个InvalidArgumentError:\newline
\includegraphics[scale=0.4]{use_gpu3.png}\newline
如果你想Tensorflow在万一指定的设备不存在时自动选择一个存在的设备,你可以在创建会话时配置中设置allow\_soft\_placement 为True
\begin{python}
with tf.device('/gpu:2'):
  a = tf.constant([1.,2.,3.,4.,5.,6.],shape= [3,2],name = 'a')
  b = tf.constant([1.,2.,3.,4.,5.,6.],shape= [2,3],name = 'b')
  c = tf.matmul(a,b)
with tf.Session(config = tf.ConfigProto(allow_soft_placement=True,log_device_placement=True)) as sess:
    print(sess.run(c))
\end{python}
\includegraphics[scale=0.4]{use_gpu4.png}\newline
用多GPU

如果你想在多张GPU上运行Tensorflow,你可以在multi-tower fashion上构造你的模型,每个tower 被指定到不同的GPU上。例如:
\begin{python}
c = []
for d in ['/gpu:0', '/gpu:1']:
    with tf.device(d):
        a = tf.constant([1.0, 2.0, 3.0, 4.0, 5.0, 6.0], shape=[2, 3])
        b = tf.constant([1.0, 2.0, 3.0, 4.0, 5.0, 6.0], shape=[3, 2])
        c.append(tf.matmul(a, b))
    with tf.device('/cpu:0'):
        sum = tf.add_n(c)
   # Creates a session with log_device_placement set to True.
sess = tf.Session(config=tf.ConfigProto(allow_soft_placement=True,log_device_placement=True))
  # Runs the op.
print(sess.run(sum))
sess.close()
\end{python}
\section{如何利用Inception的最后一层重新训练新的分类}
现代的认知模型可能有上百万个参数可能需要花几周训练,Transfer学习是通过完整的像ImageNet一样的模型通过已经存在的权重简化数周工作分类的技术。在这个例子中我们将创新训练最终层不修改其它层。详细信息你可以查看\href{http://arxiv.org/pdf/1310.1531v1.pdf}{这篇论文}.

尽管不完整的训练,但是对于一些应用却惊人的高效,可以在笔记本上训练30分钟,不要求GPU。这个导航将显示给你如何在自己的图像运行示例脚本解释一些控制训练需要的脚本。
\subsection{训练花}
\begin{center}
\begin{figure}
\includegraphics[scale=0.5]{daisies.jpg}
\end{figure}
\end{center}
在开始训练前你需要设置图像教网络你想认识的新的类别。接下来的张杰解释如何准备你的图像,但是我们创建一个授权的归档的花的文件使得训练更轻松。为了得到花的图像,运行下面的代码:
\begin{lstlisting}[language={[ANSI]C}]
cd ~
curl -O http://download.tensorflow.org/example_images/flower_photos.tgz
tar xzf flower_photos.tgz
\end{lstlisting}
当你有图像后,你从你的TensorFlow源文件目录构建重新训练器
\begin{lstlisting}[language={[ANSI]C}]
bazel build --config opt tensorflow/examples/image_retraining:retrain
\end{lstlisting}
可以通过下面运行:
\begin{lstlisting}{language=[ANSI]C}
bazel build tensorflow/examples/image_retraining:retrain
\end{lstlisting}
如果你有一个机器支持AVX设备集(最近几年的常用的x86 CPUs)你可以通过架构提高building运行速度
\begin{lstlisting}{language=[ANSI]C}
bazel build --config opt tensorflow/examples/image_retraining:retrain
\end{lstlisting}
训练器可以是这样:
\begin{lstlisting}{language=[ANSI]C}
bazel-bin/tensorflow/examples/image_retraining/retrain --image_dir ~/flower_photos
\end{lstlisting}
这个脚本载入先前incption v3模型,删除顶层,在新的flower photos训练新的模型。在原始ImageNet类中没有一种花完整网络被完整的训练过了,transfer学习是低层已经被训练好
区别不修改任何不同对象。
\subsection{瓶颈}
训练花费30分钟甚至更长时间取决于你的机器的速度。第一个时期分析所有磁盘上的图像和计算他们的瓶颈,瓶颈是一个信息对术语
我们经常在最后一层前一层,倒数第二层已经训练区别输出要求分类的值,这意味着这必须是有意义的,因此对于分类器它必须包含足够的信息在一些小的值得集合中做选择,这意味着我们的最终层训练可以在新的类中工作证明在ImageNet中1000类对于区别新的对象是有用的。
因为每个图像在训练和计算花费时间瓶颈时被多次使用,他的速度达到缓存起的瓶颈因此不能被重复计算。默认他们存储在/tmp/bottleneck陌路,如果你仍然会脚本他们将被重用,因此你不是必须再次等待这部分。
\subsection{训练}
当瓶颈计算完成时,实际顶层训练开始。你讲看到输出,显示精度,可用精度,交叉熵。训练精度显示在当前训练批中多少被分类正确,验证训练精度从图像数据集随机选中精度的值,不同之处在于训练精度基于网络已经学习到的参数,在训练中可能过拟合到为噪声。验证精确度用不在训练集中的数据性能测量精确度,如果训练精确度很高,测试精确度很低说明网络过拟合训练图像存储的部分参数没有用。交叉熵损失函数查看学习进程处理的增么样,训练对象使得损失尽可能小,因此你可以分辨出如果学习起作用,忽略损失噪声损失保持下降的趋势。默认脚本运行4000步,每一步从训练集中随机选择10张图像找到缓冲器的瓶颈,输入数据仅最终层预测。预测然后比较实际label和真实值差距反向传递误差。当你继续的时候你应该看到精确度的提高。你应该能看到精确度在90\%到95\%之间,通过提取值将随机的一次次训练,这个数完全训练好模型后基于在给定测试集中正确标签的百分比。
\subsection{用TensorBoard可视化}
包含TensorBaord总结的脚本吗很容易理解,调试,优化。例如,你可以可视化图和统计,例如在训练中权重和精度变化:
\begin{lstlisting}{language=[ANSI]C}
tensorbaord --logdir /tmp/retrain_logs
\end{lstlisting}
TensorBoard运行后导航到localhost:6006查看TensorBoard,脚本将默认采集TensorBoard总结到/tmp/retain\_logs,你可以通过summaries\_dir标志指定采集目录。
\subsection{用重新训练的模型}
脚本将用训练好的最后一层写出你的一个Inception v3版本到/tmp/output\_graph,text文件在/tmp/output\_labels.txt包含标签,两个格式见
\href{https://www.tensorflow.org/tutorials/image_recognition}{C++ and Python image classification examples},因此你可以立即开始新的模型。你去带最顶层,你将需要在脚本中指定新的名字,例如你用label\_image,用--outout\_layer=final\_result.
你可以用下面的代码重新训练图:
\begin{lstlisting}{language=[ANSI]C}
bazel build tensorflow/examples/image\_retraining:label\_image && \
bazel-bin/tensorflow/examples/image_retraining/label_image \
--graph=/tmp/output_graph.pb --labels=/tmp/output\_labels.txt \
--image=$Home/flower_photos/daisy/21625746_cc379eea_m.jpg
\end{lstlisting}
你应该看到花的标签的列表在大说书情况下daisy在顶层(尽管重新训练的模型被可能会一点不同),你可以用在你的图片上--image参数
用c++代码作为模板整合你自己的应用。
如果你想在自己的Python程序中用你训练好的模型,上面的\href{https://www.github.com/tensorflow/tensorflow/blob/r1.3/tensorflow/examples/image_retraining/label_image.py}{label\_image script}
如果你发现默认的Inception v3模型对你的应用太大或者太慢,看看\href{https://www.tensorflow.org/tutorials/image_retraining#other_model_architectures}{Other Model Architechtures section}
\subsection{在你自己的分类上训练}
如果你已经成功的让脚本在花的例子上工作,你可以教他认识其他你想它认识的东西。理论上你需要设置一个子文件夹,命名分类,每个文件夹包含分类的图像。如果你传递子文件夹的根文件夹作为参数给--image\_dir,脚本像上面训练花一样训练。
\begin{center}
\begin{figure}
\includegraphics[scale=0.5]{folder_structure.png}
\end{figure}
\end{center}
实际上它会花一点时间得到你想要的精度,下面是一些常见的问题。
\subsection{创建一个训练图像集合}
首先我们需要查看收集到的图像,常见的问题是训练过程中数据的输入。

为了训练能起作用,每个你想要识别的图像你必须至少手机100张图片,你收集到的图片越多,训练的精确度可能越好。例如你拍摄一些蓝色的房间,另一些是绿色的房间模型的预测最终基于背景颜色,没有对象特征被考虑。为了避免这种情况,拍摄不同颜色的,没有一些实际能看到的的特征。如果你想了解更多这类问题你需要读\href{http://www.jefftk.com/p/detecting-tanks}{tank recognition}
如果你想考虑你用的分类。分隔大的数据集发现一些不同的物理形式为小的可以通过视觉区分的数据集,例如你可以用'vehicle'可以用来替代
'car','motobike'和'truck',考虑你有一个开放的世界还是封闭的世界将是很有价值的,在封闭的世界你唯一需要考虑的是识别已有的对象,例如一个植物识别的app你应该知道用户可能拍摄的花的图片,英雌你必须决定花的种类,相比之下一个巡逻机器人可能通过摄像头看到不同的事物。在这种情况下你想要分类器报告是否确认他看到的,这可能很难,但是你经常收集一些典型的和主体对象不相关的背景图像,,你可能会让它增加一些图片文件夹中未知的分类。
检查确保你的图像被正确的标记也是很重要的。经常用生成的标签对于你的目的来说是不可靠的,例如你用\#daisy命名一个叫Daisy的人。如果你想你的图像如果你了解你的图像,扫除任何错误将可能导致最后精确度提高。
\subsection{训练步骤}
如果你为你的图片感到高兴,你可以通过修改学习进程中的细节提升你的结果。最简单的方法是用--how\_many\_training\_steps。默认是4000.但是如果你增加到8000,他的训练时间将增加到两倍。精确度提高的比率显示你训练的越长一些点将停止,但是你可以试验什么时候达到你的模型的限制。
\subsection{扭曲}
随机通过变形,剪裁,变化输入图像的亮度是一个提高结果的常用方法,这样扩展了训练数据的大小,帮助网络学习
真是生活分类器所有的扭曲,在脚本中使用扭曲最大的缺点是缓冲瓶颈不再有用,因此输入图像将不能重用。这意味着训练京城可能花费更多时间,因此我推荐当着作为一个调节方法调节你的模型到合理。
你可以传递--random\_corp,--random\_scale和--random\_brightness给脚本扭曲图片。百分比值用来控制图片上扭曲用多少部分。,合理的值时5或者10.--flip\_left\_right将在水平方向随机的镜像图像的一半,有助有你的应用能理解翻转的图像。例如如果你想识别字母这将不是一个好的办法,因为翻转它们会毁掉原来的含义。
\subsection{超参数}
你可以调整一些参数查看是否对你的结果有帮助,--learning\_rate控制最终层训练更新的幅度。直观理解,如果这个值变小训练时间将边长,但是他可能对精度有帮助,你需要小心试验得到查看什么对于你的case生效了。--train\_batch\_size控制每一训练步多少图像被检查,因为学习率应用到每批上,如果你有更大的批得到相同的效果你将需要减小它。
\subsection{训练,验证,测试集}
当你为你的脚本指定图像文件夹时,文件夹被分成不同的数据集。最大的数据及是训练集,训练集包含用于训练网络的数据,用于更新权重。你也许很想知道为什么我们不用所有的图像训练?一个道德潜在的问题是当我们做机器学习算法时我们的模型会记住接近正确答案的不相关信息,你可以想想你的图像可能记住了一些照片的背景,通过标签匹配对象,它在训练时所有的图像可能产生一个好的结果,但是不能再新的图像产生好的结果因为他不能泛化对象的特征,仅仅在训练图像的时候记住了一些不重要的特征。
这个问题被称为过拟合,为了避免过拟合我们保持我们的一些数据不再训练进程中,因此模型不能记住他们,我们用这些图像作为检察确保过拟合没有发生,当我们在看模型在这些数据上有一个好的精度说明过拟合没有发生,通常80\%的数据被用来作为训练集10\%的数据集用来验证最后10\%的数据用做测试集预测分类器在真实世界的性能,通过--testing\_percentage和--validation\_percentage标志用来控制比例。通常你应该能留下一些值作为默认,不应该找到任何好处训练调整他们。
注意这个脚本用图象的文件名区分训练集,验证集,测试集中的图像(不是一个随机的函数),这样保证运行时图片不会再训练集和测试集之间移动,因为当用于训练模型的图像被验证集中的图像取代时可能会出现一些问题。
你也需注意到了在迭代过程中验证正确度的波动。多数波动是验证集的子集的随机性引起的,选择的验证集用来验证精确度。波动能能被最大层度减少,花费的训练时间增长,通过选择--validation\_batch\_size=-1用整个验证集计算精度。
当训练结束后你将能检查测试集中错误分类图像,这可以通过增加--print\_misclassified\_test\_images标记,这对于找到那些什么类型的图片让模型困惑(很难区别的)是很有帮助的例如你也许发现了一些种类一些常见的图像角度是特别难是别的,这样是鼓励你增加更多类型的分类训练子类,检查催哦五分类图片也指出输入数据中的错误,向错误标签,其质量魔术的照片。然而,你应该避免测试集固定点单个误差,因为他们仅仅反映在训练集中更多的问题。
\subsection{更对模型架构}
这个脚本默认用Inception v3模型架构作为预先训练脚本。这是一个好的开始的地方,因为它提供了高精度的训练结果,但是如果你想部署你的模型到手机设备或者其他的资源限制的环境你也许想要这种精确度换区更小的文件尺寸和更快的速度。
为了帮助这个\href{https://github.com/tensorflow/tensorflow/blob/master/tensorflow/examples/image_retraining/retrain.py}{retrain}
在\href{https://research.googleblog.com/2017/06/mobilenets-open-source-models-for.html}{移动架构}上支持30个不同的变量。
这里有一些比Inception v3更小精度的,但是可以得到更小的文件大小(下载小于兆字节)运行快乐几倍。为了训练这个模型,传递--architecture标志,例如:
\begin{lstlisting}{language={ANSI}C}
python tensorflow/examples/image_retraining/retrain.py \
    --image_dir ~/flower_photos -- architecture mobilenet_0.25_128_quantized
\end{lstlisting}
这将在/temp/创建一个941KB模型文件output\_graphi.pb.Mobilenet的25\%的参数,占据$128\times128$大小的输入图像,权重在磁盘中量化为8位,你可以选择'1.0','0.75','0.50','0.25'控制权重参数的数量,因此文件尺寸(和一些扩展速度),'224','192','160'或者'128'对于输入图像的尺寸,更小的尺寸更快的速度,选项'\_quantized'预示着是否文件应该包含8位或者32位浮点权重。速度和大小好处带来的是精确度的损失,但是对于一些用途来说是不重要的,他可以通过训练数据提高、例如用扭曲在花数据集允许你得到得到80\%的精度,甚至0.25/128、quantized图。
如果你在你的程序或者label\_image中用Mobilenet模型,你讲需要一个输入一个指定大小的图像转换一个浮点让位到'input' tensor,典型的24位乳香范围[0,255]你必须用(image-128.)/128转化它到[-1,1]范围。
\section{TF layer想到:建立一个卷积神经网络}
TensorFlow\href{https://www.tensorflow.org/api_docs/python/tf/layers}{layers module}是一个用于轻松建立神经网络的高级API,它提供了一个方法促进创建dense(全连接)层和卷积层,增加激活函数,应用dropout规则。在这个导航中,你讲学习如何用layers建立一个卷积神经网络模型识别手写体数据集。
\textbf{手写体数据集包含0-9,60000个训练样本10000个测试样本,图像格式为}$28\times28$
\subsection{开始}
创建文件cnn\_mnist.py,在手写体程序中添加如下代码:
\begin{python}
from __future__ import absolute_import
from __future__ import division
from __future__ import print_function

# Imports
import numpy as np
import tensorflow as tf

tf.logging.set_verbosity(tf.logging.INFO)

# Our application logic will be added here

if __name__ == "__main__":
  tf.app.run()
\end{python}
正如你看到的,你讲增加到吗构造,训练,评估卷积神经网络,最终代码可以点击\href{https://www.github.com/tensorflow/tensorflow/blob/r1.3/tensorflow/examples/tutorials/layers/cnn_mnist.py}{这里}
\subsection{介绍卷积神经网络}
卷积神经网络是当前最先进的用于图像分类任务的模型架构。CNNsCNNs应用一些滤波器从原始的图像像素中提取高级特征,这个模型可能被用在分类。CNN包含三个组件:
\begin{itemize}
  \item \textbf{卷积层} 应用指定数量的卷积滤波器在图像上。对于每一个子区域,layer执行一系列数学操作生成一个单个值在输出feature map,卷积层然后应用relu激活函数输出非线性。
  \item \textbf{池化层} 下采样卷积层的图像数据,减小feature map的维度从而减小处理时间。常用池化算法是最大池化(提取feature map子区域)保留最大值,丢掉其它值。
  \item \textbf{Dense layers(全连接层)}在通过卷积层和下采样层特征提取执行分类。在全连接层,每一个节点连接到前面的节点。
\end{itemize}
通常CNN有一个卷积模块组成,每个层有卷积模块和池化模块组成。最新的卷积模块有一个或者更多的全连接层链接执行分类。最终CNN的全连接层包含每个目标类的一个单个节点(所有模型可能预测的类),用sofymax函数生成一个0-1的值(所有值的和维1)。我们可以解释给定图像和目标的相似情况。
\subsection{建立CNN MNIST分类器}
用CNN架构建立模型分类MNIST数据集。
\begin{enumerate}
  \item 卷积层:应用
\end{enumerate}
