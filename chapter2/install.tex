\section{安装TensorFlow for Java}
Tensorflow提供在Java程序中会死用的API。这些API特别适合载入Python创建的模型结合Java应用执行它。这个导航解释如何安装\href{https://www.tensorflow.org/api_docs/java/reference/org/tensorflow/package-summary}{TensorFlow for Java}在Java程序中使用。
\begin{quote}
\textbf{警告}TensorFlow Java API没有副高TensorFlow\href{https://www.tensorflow.org/programmers_guide/version_semantics}{API stability guarantees}
\end{quote}
\subsection{支持平台}
TensorFlow for Java支持如下操作系统:
\begin{itemize}
\item Linux
\item Mac OS X
\item Windows
\item Android
\end{itemize}
Android的安装说明在\href{https://www.github.com/tensorflow/tensorflow/blob/r1.4/tensorflow/contrib/android}{Android TensorFlow Support page}安装之后,查看在Android上TensorFlow的\ref{https://www.github.com/tensorflow/tensorflow/blob/r1.4/tensorflow/examples/android}{完整例子}
\subsection{结合Maven工程使用TensorFlow}
如果你使用\href{https://maven.apache.org/}{Apache},然后添加下面到工程的pom.xml使用TensorFlow Java APIs:
\begin{xmlcode}
<dependency>
  <groupId>org.tensorflow</groupId>
  <artifactId>tensorflow</artifactId>
  <version>1.4.1</version>
</dependency>
\end{xmlcode}
\subsection{例子}
下面是用TensorFlow创建一个Maven工程的例子:
\begin{enumerate}
\item 创建工程的pom.xml
\begin{xmlcode}
<project>
     <modelVersion>4.0.0</modelVersion>
     <groupId>org.myorg</groupId>
     <artifactId>hellotf</artifactId>
     <version>1.0-SNAPSHOT</version>
     <properties>
       <exec.mainClass>HelloTF</exec.mainClass>
       <!-- The sample code requires at least JDK 1.7. -->
       <!-- The maven compiler plugin defaults to a lower version -->
       <maven.compiler.source>1.7</maven.compiler.source>
       <maven.compiler.target>1.7</maven.compiler.target>
     </properties>
     <dependencies>
       <dependency>
         <groupId>org.tensorflow</groupId>
         <artifactId>tensorflow</artifactId>
         <version>1.4.1</version>
       </dependency>
     </dependencies>
 </project>
\end{xmlcode}
\item 创建源文件(src/main/java/HelloTF.java)
\begin{javacode}
import org.tensorflow.Graph;
import org.tensorflow.Session;
import org.tensorflow.Tensor;
import org.tensorflow.TensorFlow;

public class HelloTF {
  public static void main(String[] args) throws Exception {
    try (Graph g = new Graph()) {
      final String value = "Hello from " + TensorFlow.version();

      // Construct the computation graph with a single operation, a constant
      // named "MyConst" with a value "value".
      try (Tensor t = Tensor.create(value.getBytes("UTF-8"))) {
        // The Java API doesn't yet include convenience functions for adding operations.
        g.opBuilder("Const", "MyConst").setAttr("dtype", t.dataType()).setAttr("value", t).build();
      }

      // Execute the "MyConst" operation in a Session.
      try (Session s = new Session(g);
           Tensor output = s.runner().fetch("MyConst").run().get(0)) {
        System.out.println(new String(output.bytesValue(), "UTF-8"));
      }
    }
  }
}
\end{javacode}
\item 编译和执行:
\begin{bashcode}
 # Use -q to hide logging from the mvn tool
 mvn -q compile exec:java
\end{bashcode}
\end{enumerate}
预处理命令将从version输出Hello,如果输出,你讲成功为Java设置了TensorFlow并且准备好了在Maven工程中使用。如果没有,查看\href{http://stackoverflow.com/questions/tagged/tensorflow}{Stack Overflow }寻找可能的解决办法。你可以跳过这个文档剩下的内容。
\subsection{结合JDK使用TensorFlow}
这个章节描述如何使用TensorFlow用例子JDK安装中的java和javac命令。如果你的工程使用Apache Maven,然后查看上面简单的说明。
\subsubsection{在Linux或者Mac OS上安装}
下面的步骤为Linux或者Mac OS上的Java安装TensorFlow:
\begin{enumerate}
\item 下载TensorFlow Java归档文件\href{https://storage.googleapis.com/tensorflow/libtensorflow/libtensorflow-1.4.1.jar}{libtensorflow.jar}
\item 决定是否仅仅在CPU上运行CPU for Java或者是结合GPU帮助。为了帮你决定,读下面的文档决定安装哪个TensorFlow:
\begin{itemize}
\item \href{https://www.tensorflow.org/install/install_linux#determine_which_tensorflow_to_install}{在Linux上安装TensorFlow}
\item \href{https://www.tensorflow.org/install/install_mac#determine_which_tensorflow_to_install}{在Mac OS上安装TensorFlow}
\end{itemize}
\item 为你的操作系统下载解压合适的Java Native Interface(JNI)文件通过下面的shell命令获取处理器支持:
\begin{bashcode}
 TF_TYPE="cpu" # Default processor is CPU. If you want GPU, set to "gpu"
 OS=$(uname -s | tr '[:upper:]' '[:lower:]')
 mkdir -p ./jni
 curl -L \
   "https://storage.googleapis.com/tensorflow/libtensorflow/libtensorflow_jni-${TF_TYPE}-${OS}-x86_64-1.4.1.tar.gz" |
   tar -xz -C ./jni
\end{bashcode}
\end{enumerate}
\subsection{在Windows上安装}
在Windows上安装TensorFlow有下面几步:
\begin{enumerate}
\item TensorFlow JavaAchive(JAR)下载\href{https://storage.googleapis.com/tensorflow/libtensorflow/libtensorflow-1.4.1.jar}{libtensorflow.jar}
\item 从\href{https://storage.googleapis.com/tensorflow/libtensorflow/libtensorflow_jni-cpu-windows-x86_64-1.4.1.zip}{TensorFlow for Java on Windows}下载合适的Java Native Interface(JNI)wenjian 
\item 解压.zip文件
\end{enumerate}
\subsection{验证安装}
在安装TensorFlow for Java后,通过输入下面的代码到你的HalloTF.java文件:
\begin{javacode}
import org.tensorflow.Graph;
import org.tensorflow.Session;
import org.tensorflow.Tensor;
import org.tensorflow.TensorFlow;

public class HelloTF {
  public static void main(String[] args) throws Exception {
    try (Graph g = new Graph()) {
      final String value = "Hello from " + TensorFlow.version();

      // Construct the computation graph with a single operation, a constant
      // named "MyConst" with a value "value".
      try (Tensor t = Tensor.create(value.getBytes("UTF-8"))) {
        // The Java API doesn't yet include convenience functions for adding operations.
        g.opBuilder("Const", "MyConst").setAttr("dtype", t.dataType()).setAttr("value", t).build();
      }

      // Execute the "MyConst" operation in a Session.
      try (Session s = new Session(g);
           Tensor output = s.runner().fetch("MyConst").run().get(0)) {
        System.out.println(new String(output.bytesValue(), "UTF-8"));
      }
    }
  }
}
\end{javacode}
使用下面的命令编译和运行HelloTF.java。
\subsection{编译}
当你使用TensorFlow编译一个java程序后,下载.jar必须成为你的classpath的一部分。例如,你可以使用-cp编译标记包含下载的.jar到你的classpath:\newline 
\bashinline{javac -cp libtensorflow-1.4.1.jar HelloTF.java}\newline
\subsection{运行}
为了执行一个依赖TensorFlow的Java程序,确保下面的两个文件在你的JVM中可用:
\begin{itemize}
\item 下载的.jar文件
\item 解压的JNI库
\end{itemize}
例如,下面的命令行在Linux和Mac OS X上执行HelloTF程序。:\newline 
\bashinline{java -cp libtensorflow-1.4.1.jar:. -Djava.library.path=./jni HelloTF}\newline
下面的命令行在Windows上执行HelloTF程序:\newline 
\bashinline{bash}{java -cp libtensorflow-1.4.1.jar;. -Djava.library.path=jni HelloTF}\newline 
如果程序打印从version打印Hello,你讲成功的安装了TensorFlow for Java准备好的使用API。如果程序输出别的,查看\href{http://stackoverflow.com/questions/tagged/tensorflow}{Stack Overflow}寻找可能的解决办法。
\subsection{执行例子}
更多的高级例子查看用在图像上识别对象的\href{https://www.github.com/tensorflow/tensorflow/blob/r1.4/tensorflow/java/src/main/java/org/tensorflow/examples/LabelImage.java}{LabelImage.java}。
\subsection{从源代码构建}
TensorFlow是开源的,你可以按照下面的说明的\href{https://github.com/tensorflow/tensorflow/blob/master/tensorflow/java/README.md}{单独文档}从TensorFlow源代码构建TensorFlow源代码
