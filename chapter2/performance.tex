\chapter{性能向导}
这个向导包含了一些很好的实践来优化你的TensorFlow代码。向导被分成下面的章节:
\begin{itemize}
\item 一般的最佳实践包括常用的多种模型和硬件
\item 优化GPU详细的技巧和GPU有关
\item CPU优化 详细的CPU特定信息
\end{itemize}
\subsection{一般的最佳实践}
最佳的实践包括下面章节:
\begin{itemize}
	\item 输入pipline优化
	\item 数据格式
	\item 常用的融合操作
	\item 从源代码构建和安装
\end{itemize}
\subsubsection{输入pipeline优化}
常用的模型从磁盘获取数据在通过网络发送数据之前提前处理它,例如模型按照下面处理JPEG图像:从磁盘载入图像,解码JPEG为tensor,剪切填充,可能还有翻转和扭曲,然后分批处理。下面的图是输入pipeline。正如GPU和其它的硬件极速器会更快,预处理数据可能成为一个瓶颈。如果输入pipeline是瓶颈可能变得很复杂。一个直接的方法是在输入pipeline后减少模型为单个操作,每秒测量样本。如果对于完整的模型和单个的模型没表中样本的差别很小不同的,输入pipeline可能是瓶颈。下面有一些其它的方法识别这个问题。
\begin{itemize}
	\item 通过\lstinline[language=Bash]{watch -n 2 nvidia-smi}检查GPU是否被完全使用。如果GPU利用没有达到80-100\%,输入pipeline可能是瓶颈。
	\item 生成时间线查看大模块空白。一个生成时间线的例子在\href{https://www.tensorflow.org/performance/xla/jit}{XLA JIT}
	\item 检查CPU使用。他可能有一个优化的pipeline和缺乏CPU循环处理pipeline
	\item 评估生产力需要和验证在这个生产力条件下磁盘使用。一些云方案有网络添加磁盘速度50MB/sec,这比机械硬盘的150/MS/sec和SATA SSD的500MB/sec,PCIe SSD的2000+
 MB/sec低得多\end{itemize}
 \subsubsection{在CPU上处理}
 防止输入pipeline在CPU上能极大地提高性能。利用CPU处理输入pipeline,GPU训练。确保预处理在CPU上,按照如下操作打包:
 \begin{lstlisting}[language=Python]
 with tf.device('/cpu:0'):
     # function to get and process images or data.
     distorted_inputs = load_and_distort_images()
 \end{lstlisting}
 如果你用tf.estimator.Estimator输入函数自动放置到CPU上。
 \subsubsection{用Dataset API}
 \href{https://www.tensorflow.org/programmers_guide/datasets}{Dataset API}防止queue\_runner作为推荐的API建立输入pipelines,这个API在TensorFlow 1.2被天际到contrib中之后将被移动到TensorFlow核心。\href{https://github.com/tensorflow/models/blob/master/tutorials/image/cifar10_estimator/cifar10_main.py}{ResNet example}(\href{https://arxiv.org/pdf/1512.03385.pdf}{arXiv:1512.03385})训练CIFAR-10说明通过tf.estimator.Estimator使用Dataset API。Dataset API用C++多线程和一些低层调用而不是限制Python多线程性能的queue\_runner。尽管用feed\_dict输入数据提供了恒高的灵活性,在多数实例中feed\_dict不能规模化的优化。然而,在单个GPU上的实例使用差别可能微不足道。用Dataset API是强雷推荐的,尝试下面的代码:
 \begin{lstlisting}[language=Python]
 # feed_dict often results in suboptimal performance when using large inputs.
sess.run(train_step, feed_dict={x: batch_xs, y_: batch_ys})
 \end{lstlisting}
 \subsubsection{用大文件}
 读大量的小文件会极大地影响I/O的性能。一个方法是通过处理输入数据为(大约100MB或者更大)的TFRecord文件得到最大得到最大的I/O性能。对于小型数据集(200MB-1GB),最好的方法是载入整个数据集到内存。文档\href{https://github.com/tensorflow/models/tree/master/slim#Data}{ Downloading and converting to TFRecord format}包含一些创建TFRecord和转化CIFAR-10数据集为TFRecords的信息。
 \subsection{数据格式}
 数据格式涉及到传给Op的Tensor的结构。下面的讨论明确4D Tensor代表图像,在TensorFlow中4维Tensor的各个部分分别代表如下:
 \begin{itemize}
 	\item N:图象的批数
 	\item H:图象的高
 	\item W:代表凸显的宽
 	\item C:代表图象的通道数,1代表黑白图像,3代表真彩图像
 \end{itemize}
 在TensorFlow中有两种命名惯例,代表两种常用的格式:
 \begin{itemize}
 	\item NCHW或者channels\_first
 	\item NHWC或者channels\_last
 \end{itemize}
 NHWC是TensorFlow默认的,NCHW是在NVIDIA GPU上用cuDNN优化后的格式。构建模块的最佳实践是结合两种格式。最简单的是在GPUs上训练然后在CPUs上推理。如果TensorFLow结合\href{https://www.tensorflow.org/performance/performance_guide#tensorflow_with_intel_mkl_dnn}{Intel MKL}优化编译的,一些操作,特别是和基于CNN模型的将被优化支持NCHW。如果不用MKL,用NCHW时一些在操作将不支持在CPU上运行。
 \subsubsection{常见的融合操作}
 融合操作结合多个操作为一个内核提高性能,在TensorFlow有一些融合操作和CLA可能提高性能时创建融合操作。下面被选择的融合操作可能极大地提高性能叶东旭被忽视。
 \subsubsection{融合批规范}
 融合批规范结合多个需要批正规化的操作为一个内核。批规范对那些建立大的操作时间的比例是一个高昂的操作。用融合规范可能导致12\%-30\%的加速。有两个常见的批操作支持融合。核心tf.layers.batch\_normalization在TensorFlow 1.3 开始添加融合
 \begin{lstlisting}[language=Python]
 bn = tf.layers.batch_normallization(input_layer,fused=True,data_format = 'NCHW')
 \end{lstlisting}
 contrib中的tf.contrib.layers.batch\_norm方法从TensorFlow1.0起也有融合选项
 \begin{lstlisting}[language=Python]
 bn = tf.contrib.layers.batch_norm(input_layer, fused=True, data_format='NCHW
 \end{lstlisting}
\subsection{从源代码构建安装}
默认TensorFlow二进制针对最广泛的硬件使得TensorFlow对于每个人都可以使用。如果用CPU进行训练或者推理,推荐结合对于使用的CPU可用的优化去编译CPU。在CPU上加速推理和训练在\href{https://www.tensorflow.org/performance/performance_guide#comparing_compiler_optimizations}{Comparing compiler}被记录。

安装优化后的TensorFlow版本,从源代码\href{Comparing compiler}{构建安装},如果有在不同的硬件平台上构建TensorFlow的需要,交叉编译对于目标平台有最高的优化。下面没了是一个用bazel为指定平台编译的例子:
\begin{lstlisting}[language=Python]
# This command optimizes for Intel’s Broadwell processor
bazel build -c opt --copt=-march="broadwell" --config=cuda //tensorflow/tools/pip_package:build_pip_package
\end{lstlisting}
\subsubsection{环境构建和安装技巧}
\begin{itemize}
	\item ./configure 要求计算兼容性在构建的时候被包含。着不影响总体性能但是影响初始化启动。在运行TensorFLow一次以后,编译的内核荣光CUDA缓存。如果用一个docker容器,数据不缓存每次TensorFlow启动时间较长。通过GPUs的最佳实践是包含\href{http://developer.nvidia.com/cuda-gpus}{compute capabilities},李如意P100:6.0,Titan X(Pascal):6.1,Titan X(Maxwell):5.2和k80:3.7。
	\item 用支持所有的目标CPU优化的gcc编译器。推荐最小的gcc版本是4.8.3。在OS X上升级最新的Xcode版本用clang版本结合Xcode。
	\item 安装TensorFlow支持的最新的稳定CUDA平台和cuDNN库。
\end{itemize}
\subsubsection{优化GPU}
这部分包含指定GPU的没有在\href{https://www.tensorflow.org/performance/performance_guide#general_best_practices}{ General best practices}被包含的技巧。在多个GPU上获取优化性能是一个挑战。一个常见的方法是数据并行。通过使用数据并行利用创建多个模型的拷贝,这些模型作为一个tower,放置一个tower在每个GPU上。每个tower在不同的小批数据上操作每个tower得到更新的变量和梯度对性能有什么影响,方法,模型的收敛。下面提供了一个在多个GPUs上放置变量和tower的概览。在下一章将详细的讨论更多用于在tower分享和更新变量复杂的方法。

最好的处理变量更新的方法是依赖模型,甚至是硬件被如何配置。一个例子是一个例子是两个系统都有NVIDIA Tesla P100s一个是用PCIe另一个用\href{http://www.nvidia.com/object/nvlink.html}{NVLink}在这种场景下对于每个系统的优化方案也许不同。对于真实世界的例子,读\href{https://www.tensorflow.org/performance/benchmarks}{benchmark}详细的设置多平台优化。下面是一个benchmark多平台和配置的总结:
\begin{itemize}
	\item Tesla K80:如果GPU在相同的PCI中线跟复杂能够用\href{https://developer.nvidia.com/gpudirect}{NVIDIA GPUDirect}Peer to Peer,然后然后放置多个相等的GPUs用于训练是最好的方法。如果GPUs不能用GPUDirect,然后放置变量在CPU上是最好的选择。
	\item Titan X(Maxwell 和Pascal),M40,P100和类似的:详细的像ResNet和InceptionV,放置变量在CPU上是优化设置,但是对于有很多变量的模型像AlexNet和VGG,用GPUs和NCCL是更好。
\end{itemize}
一场用的管理变量放置的方法是创建一个方法决定每个操作放置在哪里和通过tf.device()用方法指定设备名字。考虑一个场景是一个模型在两个GPU上训练变量被放在CPU上。则将在每个GPU上循环创建放置tower,一个习惯的设备放置方法将创建监视查看操作的Variable,VariableV2和VarHanddleOp的类型,遗失者他们被放在CPU上。所有的其他操作将放在目标GPU上,构件图将按照下面处理:
\begin{itemize}
	\item 第一次循环模型中的一个tower将为gpu:0创建。当值操作期间,通常设备防治方法将指示变量被昂在cpu:0上其他操作放在gpu:0上。
	\item 第二次循环,resue设置为True预示着变量被重用tower被放置在cpu:0上被重用所有的其他操作将被创建放置在gpu:1上
\end{itemize}
最后的结果是所有的变量放在CPU上每个GPU结合模型拷贝计算操作。下面的代码段说明两个不同的变量防治方法:一个放置变量在CPU上,一个放置变量通过GPUs。
\begin{lstlisting}[language=Python]
class GpuParamServerDeviceSetter(object):
  """Used with tf.device() to place variables on the least loaded GPU.

    A common use for this class is to pass a list of GPU devices, e.g. ['gpu:0',
    'gpu:1','gpu:2'], as ps_devices.  When each variable is placed, it will be
    placed on the least loaded gpu. All other Ops, which will be the computation
    Ops, will be placed on the worker_device.
  """

  def __init__(self, worker_device, ps_devices):
    """Initializer for GpuParamServerDeviceSetter.
    Args:
      worker_device: the device to use for computation Ops.
      ps_devices: a list of devices to use for Variable Ops. Each variable is
      assigned to the least loaded device.
    """
    self.ps_devices = ps_devices
    self.worker_device = worker_device
    self.ps_sizes = [0] * len(self.ps_devices)

  def __call__(self, op):
    if op.device:
      return op.device
    if op.type not in ['Variable', 'VariableV2', 'VarHandleOp']:
      return self.worker_device

    # Gets the least loaded ps_device
    device_index, _ = min(enumerate(self.ps_sizes), key=operator.itemgetter(1))
    device_name = self.ps_devices[device_index]
    var_size = op.outputs[0].get_shape().num_elements()
    self.ps_sizes[device_index] += var_size

    return device_name

def _create_device_setter(is_cpu_ps, worker, num_gpus):
  """Create device setter object."""
  if is_cpu_ps:
    # tf.train.replica_device_setter supports placing variables on the CPU, all
    # on one GPU, or on ps_servers defined in a cluster_spec.
    return tf.train.replica_device_setter(
        worker_device=worker, ps_device='/cpu:0', ps_tasks=1)
  else:
    gpus = ['/gpu:%d' % i for i in range(num_gpus)]
    return ParamServerDeviceSetter(worker, gpus)

# The method below is a modified snippet from the full example.
def _resnet_model_fn():
    # When set to False, variables are placed on the least loaded GPU. If set
    # to True, the variables will be placed on the CPU.
    is_cpu_ps = False

    # Loops over the number of GPUs and creates a copy ("tower") of the model on
    # each GPU.
    for i in range(num_gpus):
      worker = '/gpu:%d' % i
      # Creates a device setter used to determine where Ops are to be placed.
      device_setter = _create_device_setter(is_cpu_ps, worker, FLAGS.num_gpus)
      # Creates variables on the first loop.  On subsequent loops reuse is set
      # to True, which results in the "towers" sharing variables.
      with tf.variable_scope('resnet', reuse=bool(i != 0)):
        with tf.name_scope('tower_%d' % i) as name_scope:
          # tf.device calls the device_setter for each Op that is created.
          # device_setter returns the device the Op is to be placed on.
          with tf.device(device_setter):
            # Creates the "tower".
            _tower_fn(is_training, weight_decay, tower_features[i],
                      tower_labels[i], tower_losses, tower_gradvars,
                      tower_preds, False)
\end{lstlisting}
在不远的将来上面的代码将被用于说明意图,这将被很容易用高级方法支持更广泛的方法。这个\href{https://github.com/tensorflow/models/tree/master/tutorials/image/cifar10_estimator}{例子}将更新更新作为API扩展和进展处理多个GPU的场景。
\subsubsection{优化CPU}
包含Intel Xeon Phi的CPU,当从TensorFLow原来马构建时获得最优性能所有的说明被目标CPU支持。

超过于使用最新的说明,Intel的Intel Math Kernel Library对TensorFlow深度神经网络添加了支持。尽管名字不返券静却,这些优化经常被简单的写为MKL或者tensorFlow,\href{https://www.tensorflow.org/performance/performance_guide#tensorflow_with_intel_mkl_dnn}{TensorFlow with Intel MKL-DNN}包含了MKL优化的详细说明。

下面的两个配置通过调整线程池优化CPU性能
\begin{itemize}
\item intra\_op\_parallelism\_treads:可以用多线程并行执行将调度单个片段进线程池
\item inter\_op\_parallelism\_threads:线程池所有的节点被调度
\end{itemize}
入下面显示这些配置通过tf.ConfigProto,传递给tf.Session中的config属性。对于两个配置选项,如果他们没有设置或者设置为0将默认CPU的逻辑数。测试显示对于逻辑核数为4和到多CPU的70+是高效的。一个常用的优化是设置在池中的线程数等于物理核数而不是逻辑核数。
\begin{lstlisting}[language=Python]
config = tf.ConfigProto()
config.intra_op_parallelism_threads = 44
config.inter_op_parallelism_threads = 44
tf.session(config=config)
\end{lstlisting}
下面比较编译器优化章节包含了不同编译器优化测试结果
\subsubsection{TensorFlow和Intel MKL DNN}
Intel通过用Intel MKL-DNN为Intel Xeon和Xeon Phi添加了对TensorFlow的优化。优化对于消耗处理器行提供了加速,如i5和I5的Intel处理器。在论文\href{https://software.intel.com/en-us/articles/tensorflow-optimizations-on-modern-intel-architecture}{TensorFlow* Optimizations on Modern Intel® Architecture }包含了详细的实现。
\begin{quote}
\emph{MKL在tensorFlow 1.2就被添加,当前尽在Linux上有小。如果用config=cuda它将不工作}
\end{quote}
另外对基于CNN的模型提供了极大地性能提升,和MKL编译创建对AVXhe AVX2的优化的一个二进制。结果是一个二进制被优化兼容多数处理器(2011年以后的)。
TensorFlow可以通过下面的命令在结合MKL优化通过源代码被编译。对于TensorFlow之后的源代码版本:
\begin{lstlisting}[language=Bash]
./configure
# Pick the desired options
bazel build --config=mkl -c opt //tensorflow/tools/pip_package:build_pip_package
\end{lstlisting}
对于TensorFlow版本为到1.3.0:
\begin{lstlisting}[language=Bash]
./configure
Do you wish to build TensorFlow with MKL support? [y/N] Y
Do you wish to download MKL LIB from the web? [Y/n] Y
# Select the defaults for the rest of the options.

bazel build --config=mkl --copt="-DEIGEN_USE_VML" -c opt //tensorflow/tools/pip_package:build_pip_package

\end{lstlisting}
\subsubsection{调整MKL获得最佳性能}
这部分将介绍不同的配置和环境变量用于调整MKL得到优化的性能。在调整多环境变量之前确保模型使用NCHW(chennel\_first)数据核实。MKL对于NCHW被优化过当用NCHW时Intel将获得最高性能。
MKL用下面的环境变量调整西鞥能:
\begin{itemize}
	\item KMP\_BLOCKTIME-设置时间,毫秒,表示线程应该等待的时间在完成并行执行后,休眠之前
	\item KMP\_AFFINITY-启动运行时间库绑定线程到物理处理器单元
	\item KMP\_SETTING-在程序知识中启动或者警用OpenMP*打印运行事件库环形变量
	\item OMP\_NUM\_THREADS指定使用的线程数
\end{itemize}
更多的关于KMP变量在\href{https://software.intel.com/en-us/node/522775}{Intel's}网站上OMP变量在\href{https://gcc.gnu.org/onlinedocs/libgomp/Environment-Variables.html}{gnu.org}
尽管通过调整环境变量可能有极大的提升,下面的讨论简单的建议通过下面的环境变量设置inter\_op\_:parallelism\_threads等于物理CPU的核数用
\begin{itemize}
	\item KMP\_BLOCKTIME=0
	\item KMP\_AFFINITY=granularity=fine,verbose,compact,1,0
\end{itemize}
用命令行设置MKL环境变量:
\begin{lstlisting}[language=Bash]
KMP_BLOCKTIME=0 KMP_AFFINITY=granularity=fine,verbose,compact,1,0 \
KMP_SETTINGS=1 python your_python_script.py
\end{lstlisting}
通过Python的os.environ设置MKL环境变量:
\begin{lstlisting}[language=Python]
os.environ["KMP_BLOCKTIME"] = str(FLAGS.kmp_blocktime)
os.environ["KMP_SETTINGS"] = str(FLAGS.kmp_settings)
os.environ["KMP_AFFINITY"]= FLAGS.kmp_affinity
if FLAGS.num_intra_threads > 0:
  os.environ["OMP_NUM_THREADS"]= str(FLAGS.num_intra_threads)
\end{lstlisting}
有一些模型和硬件平台在不同的设置上得到好处。每个变量影响性能在下面被讨论:
\begin{itemize}
\item KMP\_BLOCKTIME:MKL默认为200ms,没有为我们的测试优化 0ms是一个好的默认CNN基础的测试模型AlexNet的最好性能设置为30ms,GoogleNet和VGG11最好设置为1ms。
\item KMP\_AFFINITY:推荐设置是granularity=fine,verbose,compact,1,0
\item OMP\_NUM\_THREADS:默认物理核数,当用Intel Phi在模型上是调整参数超过匹配的核数有些印象,更多的模型查看\href{https://software.intel.com/en-us/articles/tensorflow-optimizations-on-modern-intel-architecture}{TensorFlow* Optimizations on Modern Intel Architecture}获得优化性能
\item intra\_op\_parallelism\_threads:推荐设置这个等于物理核数,设置值为0默认导致值被设置逻辑核的个数,对一些架构是一个尝试选项,值和OMP\_NUM\_THREADS应该相等
\item inter\_op\_parallelism\_threads:推荐设置等于sockes的数量,默认设置值为0将导致值被设置为逻辑核数
\end{itemize}
\subsubsection{比较编译器的优化}
在不同的CPU平台上用不同的编译器优化收集训练和推理的执行信息。模型被用在ResNet-50\href{https://arxiv.org/abs/1512.03385}{arXiv:1512.03385}和Inception V3\href{https://arxiv.org/abs/1512.00567}{arXiv:1512.00567}
对于更多的测试,当在环境变量KMP\_BLOCKTIMEMKL优化被使用被设置为0ms,KMP\_AFFINITY为granularity=fine,verbose,compact,1,0

推理InceptionV3\newline
\textbf{环境}
\begin{itemize}
	\item 实例类型:AWS EC2 m4.xlarge
	\item CPU:Intel(R) Xeon(R) CPU E5-2686 v4 @ 2.30GHz (Broadwell)
	\item 数据集:Imagenet
	\item TensorFlow 版本1.2.0 RC2
	\item 测试脚本:\href{https://github.com/tensorflow/benchmarks/blob/mkl_experiment/scripts/tf_cnn_benchmarks/tf_cnn_benchmarks.py}{tf\_cnn\_benchmarks.py}
\end{itemize}
\textbf{Batch Size}
为MKL测试执行命令
\begin{lstlisting}[language=Bash]
python tf_cnn_benchmarks.py --forward_only=True --device=cpu --mkl=True \
--kmp_blocktime=0 --nodistortions --model=inception3 --data_format=NCHW \
--batch_size=1 --num_inter_threads=1 --num_intra_threads=4 \
--data_dir=<path to ImageNet TFRecords>
\end{lstlisting}

\begin{tabular}{|c|c|c|c|c|}
\hline
优化&数据格式&图像/秒(步时)&Intra线程&Inter线程\\
\hline
AVX2&NHWC&6.8(147ms)&4&0\\
\hline
MKL&NHWC&6.8(147ms)&4&1\\
\hline
MKL&NHWC&5.95(168ms)&4&1\\
\hline
AVX&NHWC&4.7(211ms)&4&0\\
\hline
SSE3&NHWC&2.7(370ms)&4&0\\
\hline
\end{tabular}

\textbf{Batch Size:32}
执行MKL测试命令:
\begin{lstlisting}[language=Bash]
python tf_cnn_benchmarks.py --forward_only=True --device=cpu --mkl=True \
--kmp_blocktime=0 --nodistortions --model=inception3 --data_format=NCHW \
--batch_size=32 --num_inter_threads=1 --num_intra_threads=4 \
--data_dir=<path to ImageNet TFRecords>
\end{lstlisting}

\begin{tabular}{|c|c|c|c|c|}
\hline
优化&数据格式&图像/秒(步时)&Intra线程&Inter线程\\
\hline
MKL&NCHW&10.24 (3125ms)	4&1\\
\hline
MKL&NHWC&	8.9 (3595ms)&4&1\\
\hline
AVX2&NHWC&7.3 (4383ms)&4&0\\
\hline
AVX&NHWC&5.1 (6275ms)&4&0\\
\hline
SSE3&NHWC&2.8 (11428ms)&4&0\\
\hline
\end{tabular}

\textbf{推理ResNet-50}
\begin{itemize}
	\item 实例类型: AWS EC2 m4.xlarge
	\item CPU: Intel(R) Xeon(R) CPU E5-2686 v4 @ 2.30GHz (Broadwell)
	\item 数据集: ImageNet
	\item TensorFlow 版本: 1.2.0 RC2
	\item 测试脚本: \href{https://github.com/tensorflow/benchmarks/blob/mkl_experiment/scripts/tf_cnn_benchmarks/tf_cnn_benchmarks.py}{tf\_cnn\_benchmarks.py}
\end{itemize}

\begin{tabular}{|c|c|c|c|c|}
优化&数据格式&图像/秒(步时)&Intra线程&Inter线程\\
\hline
AVX2&	NHWC&	6.8 (147ms)&	4&	0\\
\hline
MKL&	NCHW&	6.6 (151ms)&	4&	1\\
\hline
MKL&	NHWC&	5.95 (168ms)&	4&	1\\
\hline
AVX&	NHWC&	4.7 (211ms)&	4&	0\\
\hline
SSE3&	NHWC&	2.7 (370ms)&	4&	0\\
\hline
\end{tabular}

\textbf{Batch size:32}\newline
执行MKL测试的命令
\begin{lstlisting}[language=Python]
python tf_cnn_benchmarks.py --forward_only=True --device=cpu --mkl=True \
--kmp_blocktime=0 --nodistortions --model=resnet50 --data_format=NCHW \
--batch_size=32 --num_inter_threads=1 --num_intra_threads=4 \
--data_dir=<path to ImageNet TFRecords>
\end{lstlisting}

\begin{tabular}{|c|c|c|c|c|}
\hline
优化&数据格式&图像/秒(步时)&Intra线程&Inter线程\\
\hline
MKL&	NCHW&	10.24 (3125ms)&	4&	1\\
\hline
MKL&	NHWC&	8.9 (3595ms)&	4&	1\\
\hline
AVX2&	NHWC&	7.3 (4383ms)&	4&	0\\
\hline
AVX&	NHWC&	5.1 (6275ms)&	4&	0\\
\hline
SSE3&	NHWC&	2.8 (11428ms)&	4&	0\\
\hline
\end{tabular}

\textbf{训练InceptionV3}\newline
\textbf{环境}
\begin{itemize}
	\item 实例类型: Dedicated AWS EC2 r4.16xlarge (Broadwell)
	\item CPU: Intel Xeon E5-2686 v4 (Broadwell) Processors
	\item 数据集: ImageNet
	\item TensorFlow 版本: 1.2.0 RC2
	\item 测试脚本: \href{https://github.com/tensorflow/benchmarks/blob/mkl_experiment/scripts/tf_cnn_benchmarks/tf_cnn_benchmarks.py}{tf\_cnn\_benchmarks.py}
\end{itemize}

执行MKL测试命令
\begin{lstlisting}[language=Python]
python tf_cnn_benchmarks.py --device=cpu --mkl=True --kmp_blocktime=0 \
--nodistortions --model=resnet50 --data_format=NCHW --batch_size=32 \
--num_inter_threads=2 --num_intra_threads=36 \
--data_dir=<path to ImageNet TFRecords>
\end{lstlisting}
\begin{tabular}{|c|c|c|c|c|}
\hline
优化&数据格式&图像/秒(步时)&Intra线程&Inter线程\\
\hline
MKL&	NCHW&	20.8&	36&	2\\
\hline
AVX2&	NHWC&	6.2&	36&	0\\
\hline
AVX&	NHWC&	5.7&	36&	0\\
\hline
SSE3&	NHWC&	4.3&	36&	0\\
\hline
\end{tabular}
ResNet和AlexNet在这个噢诶纸上运行但是有一个 ad hoc行为。没有足够运行执行结合标的结果。不完整的结果强烈的暗示了最后的结果将雷士上面的MKL比AVX2大约3x+的提升。
