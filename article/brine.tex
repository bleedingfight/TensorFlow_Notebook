\documentclass{article}
\usepackage[space]{ctex}
\usepackage{listings}
\usepackage{hyperref}
\usepackage{color}
\definecolor{codegreen}{rgb}{0,0.6,0}
\definecolor{codegray}{rgb}{0.5,0.5,0.5}
\definecolor{codepurple}{rgb}{0.58,0,0.82}
\definecolor{backcolour}{rgb}{0.95,0.95,0.92}
 
\lstdefinestyle{mystyle}{
    backgroundcolor=\color{backcolour},   
    commentstyle=\color{codegreen},
    keywordstyle=\color{magenta},
    numberstyle=\tiny\color{codegray},
    stringstyle=\color{codepurple},
    basicstyle=\footnotesize,
    breakatwhitespace=false,         
    breaklines=true,                 
    captionpos=b,                    
    keepspaces=true,                 
    numbers=left,                    
    numbersep=5pt,                  
    showspaces=false,                
    showstringspaces=false,
    showtabs=false,                  
    tabsize=2
}
 
\lstset{style=mystyle}
 
\begin{document}
\section{安装}
brine是一个pip包,因此我们可以用下面的命令安装

\lstinline[language=Bash]{pip install brine-io}
\section{下载数据集}
在你的项目坐在目录运行brine命令行下载CIFAR10数据集。

\lstinline[language=Bash]{brine install cifar10/train}
\section{载入数据集}
用load\_dataset()函数载入数据集
\begin{lstlisting}[language=Python]
import brine
cifar_train = brine.load_dataset('cifar10/train')
\end{lstlisting}
用数据集的columns属性我们可以查看数据集的结构
\begin{lstlisting}
cifar_train.columns
Column(image=Column(name=image, type=Image), label=Column(name=label, type=Category, categories=['dog', 'horse', 'frog', 'airplane', 'cat', 'ship', ...]))
\end{lstlisting}
返回一个通过列的名字索引的元祖的Column,因此label是一个类别列,我们可以查看列包含的分类,下面我们将分类保存在本地变量中
\begin{lstlisting}[language=Python]
categories = cifar_train.columns.label.categories
categories
['dog', 'horse', 'frog', 'airplane', 'cat', 'ship', 'deer', 'bird', 'truck', 'automobile']
\end{lstlisting}
通过len检查数据集的长度
\begin{lstlisting}[language=Python]
>> len(cifar_train)
50000
\end{lstlisting}
从任何行访问数据集,我们简单的传递索引进去。它返回一个命名的元祖,我们可以访问单个的元素
\begin{lstlisting}[language=Python]
>> cifar_train[20]
Row(image='46405_bird.png', label='bird')
>> cifar_train[20].image
'46405_bird.png'
>> cifar_trian[20].label
'bird'
\end{lstlisting}
image返回一个图像路径。用数据集的load\_image()方法载入图片
\begin{lstlisting}[language=Python]
>> cifar_train.load_image(cifar_train[20].image)
<PIL.PngImagePlugin.PngImageFile image mode=RGB size=32x32 at 0x7F9BA7860D68>
\end{lstlisting}
我们可以用create\_folds()分割我们的数据集为多个文件夹。这用于创建train/validation分割。
brine数据集中的每个文件夹(不占据额外的磁盘空间)因此你可以在原始数据集的文件夹伤执行相同的行为
,下面分割2000个向本用于我们的验证文件夹,剩下的样本将进入训练文件夹
\begin{lstlisting}[language=Python]
>> validation_fold, training_fold = cifar_train.create_folds([2000],
shuffle=True)
>> len(validation_fold)
2000
>> len(training_fold)
48000
\end{lstlisting}
\section{PyTorch接口}
Brine提供方法转换你的Brine数据集问流行的机器学习兼容数据,像PyTorch和keras
这里是用PyTorch的例子
\begin{lstlisting}[language=Python]
from torchvision import transforms
transform = transforms.Compose(
    [transforms.ToTensor(),
    transforms.Normalize((0.5, 0.5, 0.5), (0.5, 0.5, 0.5))])

train_fold_pytorch = training_fold.to_pytorch(transform=transform, transform_columns='images')
validation_fold_pytorch = validation_fold.to_pytorch(transform=transform, transform_columns='images')
\end{lstlisting}
to\_pytorch()方法返回一个PyTorch数据集。转化调用用在数据集的每一行,因此我们指定transform\_columns='images',转化将被用于图像列(这里是image),图像列被传递前将被转化为PIL图像,你可以用这个PyTorch数据集正如我们用其它的PyTorch数据集一样,例如你可以用它创建一个DataLoader。
\lstinline[language=Python]{trainloader = torch.utils.data.DataLoader(train_fold_pytorch, batch_size=4, shuffle=True, num_workers=2)}
\section{结论}
可以用相同的方法应用通过Brine应用到任何数据集,包括分割,分类,多类分类和对象检测,用Keras结合Brine分割图像的例子查看\href{https://medium.com/@hanrelan/a-non-experts-guide-to-image-segmentation-using-deep-neural-nets-dda5022f6282}{这篇博客}
\end{document}
